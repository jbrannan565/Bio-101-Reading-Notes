\documentclass{article}
\usepackage[utf8]{inputenc}

\title{Week 1 Lecture 1}
\author{Jared Brannan }

\usepackage{natbib}
\usepackage{graphicx}
\usepackage{mathtools}
\usepackage{amsthm}
\usepackage{amsmath}
\usepackage{amssymb}
\usepackage{xcolor}
\usepackage{bbm}
\usepackage{bm}
\usepackage{physics}

% indent first line
\usepackage{indentfirst}

\theoremstyle{definition}

\newcommand{\upRiemannint}[2]{
\overline{\int_{#1}^{#2}}
}
\newcommand{\loRiemannint}[2]{
\underline{\int_{#1}^{#2}}
}

\newtheorem{definition}{Definition}
\newtheorem{asside}{Asside}
\newtheorem{conjecture}{Conjecture}
\newtheorem{example}{Example}
\newtheorem{theorem}{Theorem}
\newtheorem{lemma}{Lemma}
\newtheorem{puzzle}{Puzzle}
\newtheorem{corollary}{Corollary}
\newtheorem{proposition}{Proposition}


\begin{document}
\maketitle

We'll pick scientists on friday -- have 3 in mind.

\section{Characteristics of life -- continued...}

\subsection{Homeostasis continuted...}

- Homeostasis is mainly maintained via negative feedback systems.

- Maintain a constant \textit{internal} environment in the eface of variable \textit{external} environment.

- This takes energy, hence, work.

- How do we maintain body temperature? What could you do if the external environment is too hot or too cold?

\indent\indent - there are seperate systems for each

- \textbf{Feedback systems} help maintain homeostasis and are (broadly):

\indent\indent- [usually negative]

\indent\indent - \textit{stimulus receptor} monitors the environment (e.g. the skin)

\indent\indent - \textit{Control center}  recieves info and decides what to do (e.g. brain)

\indent\indent - \textit{Effector}  receives instructions and carries out response (e.g. muscles contract)

- almost all heat created by the body comes from muscle contraction

- the feedback system(s) for cooling work in tandem with the system for heating.

- similar systems: thermostat or cruise control.

- possitive feedbacks lead to extremes, while negative stabalizes\\~\\

- 3 ways to adjust the internal env:

\indent\indent - move to a more favorable env

\indent\indent - later your env to be more favorable

\indent\indent - Your body's \textit{homeostatic}  controll systems can compensate for the dif between ideal internal and poor external conditions

- maintiaining sensory systems and enacting the response mechanism both have energenic costs

- if not being used for homeostasis, that energy could have been used for other purposes.

\subsection{Summary}

- 9 characteristics of life 

- Homeostasis maintains those 9 characteristics.

\section{Characteristics of Kingdom Animalia}

- Who's related to whom?

- there are approximately 10 million multicellular speces

- most multicellular life forms are insects.

- Classifying animals is done using a heirarchy, `kingdoms' is rather large
-- domain - kingdom - phyla
\\~\\


Humans are in \textbf{Kingdom Animalia}, so we are/have:

\indent\indent- Animals

\indent\indent- Multiicellular

\indent\indent- \textit{Heterotrophic} == we get what we need from the food we eat. Energy always comes from plants, which get it from the sun

\indent\indent- Lack rigid cell walls -- this allows us to carry out complex motion\\~\\

- plants are \textit{autotrophic} -- self feeding

- plants and fungi have cells with ridgid walls\\~\\

There are ~35 animal \textbf{phyla}:

\indent\indent - 34: sponges, worms, insects, molluscs, jellyfish,...

\indent\indent - phylum \textbf{Chordata} -- where humans are. The closest recent ansestor between chordates and everything else is the sea urchin (or, its ansestor).

Characteristics of \textbf{phylum chrodata}:

\indent\indent - \textit{notochord} for support (verts. spinal col)

\indent\indent - Dorsal, hollow \textit{nerve chord} (spinal cord)

\indent\indent - \textit{Pharyngeal gill slits/pouches} (tetrapoda, only in ebryo. There's a rare mutation where human adults have the slits, but usually they are removed during development) 

\indent\indent - \textit{Post-anal tail}  (in humans, coccyx) (this also means we have a digestive system with an exit. i.e. an anus)


\end{document}
