\documentclass{article}
\usepackage[utf8]{inputenc}

\title{Week 5 Lecture 2}
\author{Jared Brannan }

\usepackage{natbib}
\usepackage{graphicx}
\usepackage{mathtools}
\usepackage{amsthm}
\usepackage{amsmath}
\usepackage{amssymb}
\usepackage{xcolor}
\usepackage{bbm}
\usepackage{bm}
\usepackage{physics}

% indent first line
\usepackage{indentfirst}
% one inch margins
\usepackage[margin=1.0in]{geometry}

\theoremstyle{definition}

\newcommand{\upRiemannint}[2]{
\overline{\int_{#1}^{#2}}
}
\newcommand{\loRiemannint}[2]{
\underline{\int_{#1}^{#2}}
}

\newtheorem{definition}{Definition}
\newtheorem{asside}{Asside}
\newtheorem{conjecture}{Conjecture}
\newtheorem{example}{Example}
\newtheorem{theorem}{Theorem}
\newtheorem{lemma}{Lemma}
\newtheorem{puzzle}{Puzzle}
\newtheorem{corollary}{Corollary}
\newtheorem{proposition}{Proposition}


\begin{document}
\maketitle

\section{Administrative drivel}
\begin{itemize}
	\item
\end{itemize}

\section{More on Cells}
\begin{itemize}
	\item More on organelles
		\begin{itemize}
			\item New protiens are sent to the golgi aperatus in vesicles from the RER and SER
				\begin{itemize}
					\item Vesicle: a pocket of membrane that is created by exocytosis from the ER.
				\end{itemize}
			\item the vesicles fuse with the membrane of the golgi aperatus, and the proteins and lipids are modified.
		\end{itemize}
	\item \textbf{Golgi complex:}
		\begin{itemize}
			\item Parts: Saccule/cistern, transfer vesicle
				\begin{itemize}
					\item multiple stacked layers of cisterns (pockets)
				\end{itemize}
			\item vesicles enter into the back into a saccule, and enzymes work on the lipids and polypeptides
			\item they work their way to the front of the golgi complex where they're fully formed, and moved into a vesicle via exocytosis from a cistern.
			\item There's never an opening to allow things to leak into the containers of the golgi aperatus
			\item the proteins are then sent off to do work
		\end{itemize}
	\item \textbf{Mitochondria:}
		\begin{itemize}
			\item  The powerhouse of the cell
			\item 2 membranes: outer and inner
				\begin{itemize}
					\item inner has folds that greatly increase the internal surface area
					\item The presense of the double membrane was what lead to the hypothesis that the mitochondria was an independent organism brought in via endocytosis, and it's involved to be part of the cell
				\end{itemize}
			\item reproduce independently of the host, and have their own DNA
			\item function: generate ATP
				\begin{itemize}
					\item takes in glucose and makes 2 ATP per glucose even in the absence of oxygen, but oxygen allows for creating many more ATP per glucose.
					\item note O2 is not an absolute requirement for ATP synthesis
					\item steps:
						\begin{itemize}
							\item Take glucose and break the carbon-carbon bonds, and capture the energy as an excited electron
							\item this is used to combine an ADP and a phosphate
							\item the glucose is split into 2 pruvic acids, making 2 atp
								\begin{itemize}
									\item this is fermintation!
									\item these acids are high energy!
								\end{itemize}
							\item the acids are broken down (using oxygen) till all of the carbon is turned into CO2 and all of the hydroggen and oxygen become water
							\item the end result is usually 36 ATP (including the non-arrobic initial 2)
						\end{itemize}
				\end{itemize}
			\item this is generally the only place where ATP is made
			\item basic equation:
				\[
				C_6H_{12}O_6 + 6O_2\to 6CO_2 + 6H_{20} + 36 ATP
				.\] 
			\item this equation is exactly the reverse of photosynthesis! :D
			\item earliest organisms were syonobacteria 2.7 billion ish years ago
				\begin{itemize}
					\item all of the oxygen were bound to rocks
					\item these bacteria developed photosynthesis and turned the oxygen in the rocks into gaseous oxygen! Now mostly plants make the bulk of the oxygen
					\item Nitrogen is the most abundent gas in the atmosphere, then oxygen, and everything else makes up less than 1 percent.
					\item making ATP using other minerals would have been before photosynthesis for these reasons
				\end{itemize}
			\item Clicker Q: When DNA sequence is packaged into chromosomes, what other category of molecule is also present? Protein! make up the histones
		\end{itemize}
	\item summary:
		\begin{itemize}
			\item Nuclues: contains DNA, the largest molecule in the cell as chromosomes
			\item RER and SER where ribosomes read mRNA to make polypeptides, and fatty acids are made respectively
			\item Golgi aperatus: where protiens and lipids are adjusted
			\item Mitochondria: where ATP is made
		\end{itemize}
	\item Clicker Q: What is an organelle? Structures in cells that have specialized functions
\end{itemize}

\section{Tissues}
Made up of cells!
\begin{itemize}
	\item Tissues are aggragates of cells (usually of the same type)
	\item tissues are organized into organs
	\item 4 types of tissues:
		\begin{itemize}
			\item Epithelial
				\begin{itemize}
					\item Outer layer (not just skinn)
					\item packed tight
					\item dense sheet of closely packed cells attached to an underlying non-cellular layer (basement membrane)
						\begin{itemize}
							\item usualy 1 celll thick
							\item this basement membrane is an excreted structure made of proteins and other things.
						\end{itemize}
					\item lung cells that line the lungs
					\item cells lining the digestive system
						\begin{itemize}
							\item mouth, esophogus, etc.
						\end{itemize}
					\item cells lining blood vessels
					\item cells lining urinary and reproductive structures
					\item also kidney and liver cells
						\begin{itemize}
							\item Most of these are epithelial
						\end{itemize}
					\iteme they tend to cover organs and linne tube and cavity surfaces as a protective layer against things like micro-organisms.
					\item they also regulate what comes in and out oof the body
					\item they are the first line of defense - skin, and other layers protects against microorganisms
					\item secretes proteins,e.g. karetin on the skin which microorganisms don't like to eat
						\begin{itemize}
							\item e.g. digestive tract produces mucus (protein and carbohydrate slimee) that coats the digestive tract preventing enzymes and digestive acid from eating your organs
						\end{itemize}
					\item has it's own sweet of carrier protiens and pumps to regulate what goes through.
					\item Control what goes through into underlying layers
						\begin{itemize}
							\item e.g. the digestive system prevents toxins from getting in
						\end{itemize}
					\item Takes up specific ions/ other types of molecules, eg.g kidneys, digestive tract, liver
					\item e.g. Smalll intestine epithelial cells
						\begin{itemize}
							\item glandular cells, secrete mucus into the intestine helps stuff move throuogh smoothly and prevent degestive enzymes from breaking down the intestinal wall. E-cells in the intestine take up broken-down carbs, lipids, and peptides
						\end{itemize}
					\item since they secrete things, they also make up glands
				\end{itemize}
			\item Connective tissue/extracellur matrix
				\begin{itemize}
					\item 
				\end{itemize}
		\end{itemize}
\end{itemize}



\end{document}
