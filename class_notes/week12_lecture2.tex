\documentclass{article}
\usepackage[utf8]{inputenc}

\title{Week 12 Lecture 2}
\author{Jared Brannan }

\usepackage{natbib}
\usepackage{graphicx}
\graphicspath{ {./figures/} }
\usepackage{mathtools}
\usepackage{amsthm}
\usepackage{amsmath}
\usepackage{amssymb}
\usepackage{xcolor}
\usepackage{bbm}
\usepackage{bm}
\usepackage{physics}

% indent first line
\usepackage{indentfirst}
% one inch margins
\usepackage[margin=1.0in]{geometry}

\theoremstyle{definition}

\newcommand{\upRiemannint}[2]{
\overline{\int_{#1}^{#2}}
}
\newcommand{\loRiemannint}[2]{
\underline{\int_{#1}^{#2}}
}

\newtheorem{definition}{Definition}
\newtheorem{asside}{Asside}
\newtheorem{conjecture}{Conjecture}
\newtheorem{example}{Example}
\newtheorem{theorem}{Theorem}
\newtheorem{lemma}{Lemma}
\newtheorem{puzzle}{Puzzle}
\newtheorem{corollary}{Corollary}
\newtheorem{proposition}{Proposition}


\begin{document}
\maketitle

\section{Administrative drivel}
\begin{itemize}
	\item I missed class wednesday -- lecture notes to be made; I have slides to go off of.
	\item still waiting on exam scores from electronic grading
	\item we might get a review session before the final (depending on profs productivity over break)
		\begin{itemize}
			\item answering questions brought in
		\end{itemize}
	\item there will be a review sheet for the final exam
\end{itemize}

\section{Diseases}
\subsection{protozoal diseases}

\begin{itemize}
	\item malaria -- 2nd most common disease that kills a significant number (on order of 1 million)
		\begin{itemize}
			\item \textit{Plasmodium} (the organism) lives in mosquito salivary glands
			\item mosquitos use saliva to anaesthetize their host (so they don't feel the bite)
				\begin{itemize}
					\item Almost always a female
					\item they largely eat nectar from flowers, which is low in protein, so the females need blood for protien to make eggs
					\item blood is high water volume, with some protien
					\item lifecycle of plasmodium:
						\begin{itemize}
							\item plasmodium is taken up into mosquito stomach
							\item makes its way into the salivary glands
							\item Gets injected into new host via salivary glands
							\item eventually destroys red blood cells, entering blood plasma in the millions
							\item some communication between plasmodia
							\item plasmodia have (usually) 4 day life cycles
						\end{itemize}
					\item immune system isn't great at removing it
					\item some defense can be mounted, but in children under 5 the defenses are minimal, so most children die
					\item every 4 days you're body shuts down (from the red blood cells ruptuing)
				\end{itemize}
			\item severe flu-like symptoms
			\item 2nd only to tb in terms of deaths
			\item largely irraticated from the US
			\item is actually a family of organisms, diffeerent ones infecting different  species
			\item Eukareotic, so looks like our cells... So, the drugs we have attack our own cells as well as the pathogen
			\item 2015 WHO data:
				\begin{itemize}
					\item more than 200 million cases 
					\item more than 400k deaths, mostly children in Africa
				\end{itemize}
		\end{itemize}
	\item prevention:
		\begin{itemize}
			\item Dirnk only clean water
			\item Avoid organisms that pass the protozoa to you
		\end{itemize}
	\item Protozoa are extremely diverse, and so othere is no one class of drug that can treat them all
	\item each group of protoa is treated with its own anti-protozoal drug
	\item They are eukaryotes, like us -- much harder to find unique biochemistry
	\item E.g. in US: Giardia!
	\item E.g. Tremetodes and flat worms
		\begin{itemize}
			\item Many attack the nervous system, liver
			\item Next to impossible to be rid of them
		\end{itemize}
\end{itemize}
\subsection{Other infectious diseases}
\begin{itemize}
	\item Fungi -- ringworm, athlete's foot, thrush, yeast infection
	\item Parasitic worms -- pinworms, hookworms, tapeworms, flukes (the last cause severe symptoms)
	\item Prions (non living)  -- infectious \textit{protiens} mad cow (bovin encephalitis), jakob creutzfeldt disease, CWD (ungulates as far as we know)
\end{itemize}

\section{Senses}
\begin{itemize}
	\item  What is a sense?
		\begin{itemize}
			\item Systems where our bodies are ablee to detect things that are going on in the enviourmnent (usually detection of some form of eenergy)
			\item A distinct way we recieve stimuli
				\begin{itemize}
					\item (a signal from the nearby enviornment)
					\item 1. A \textbf{stimulus} arives:
						\begin{itemize}
							\item A molecule enters the nose
							\item sound wwave enters the ear
							\item you touch the tabletop
						\end{itemize}
					\item 2. The stimulus \textit{activates} a \textbf{receptor cell} 
						\begin{itemize}
							\item the receptor cell's main job is to wait for a \underline{specific} stimulus
						\end{itemize}
					\item 3. And then the receptorc cell
						\begin{itemize}
							\item communicates with the \textbf{nervous system} 
								\begin{itemize}
									\item Not always the brain
									\item e.g. reflex networks
									\item ganglia (mini brains) in the spinal chord
								\end{itemize}
							\item nervous system decides what to do about the stimulus
						\end{itemize}
				\end{itemize}
		\end{itemize}
	\item Usually some sort of negative feedback system
		\begin{itemize}
			\item stimulus receptor
			\item control center
			\item effector
		\end{itemize}
	\item Sensory systems connect to nervous tissues, which:
		\begin{itemize}
			\item sense the enviornment
			\item organize the appropriate response
			\item keep things ticking along
		\end{itemize}
\end{itemize}

\end{document}
