\documentclass{article}
\usepackage[utf8]{inputenc}

\title{Week 11 Lecture 1}
\author{Jared Brannan }

\usepackage{natbib}
\usepackage{graphicx}
\graphicspath{ {./figures/} }
\usepackage{mathtools}
\usepackage{amsthm}
\usepackage{amsmath}
\usepackage{amssymb}
\usepackage{xcolor}
\usepackage{bbm}
\usepackage{bm}
\usepackage{physics}

% indent first line
\usepackage{indentfirst}
% one inch margins
\usepackage[margin=1.0in]{geometry}

\theoremstyle{definition}

\newcommand{\upRiemannint}[2]{
\overline{\int_{#1}^{#2}}
}
\newcommand{\loRiemannint}[2]{
\underline{\int_{#1}^{#2}}
}

\newtheorem{definition}{Definition}
\newtheorem{asside}{Asside}
\newtheorem{conjecture}{Conjecture}
\newtheorem{example}{Example}
\newtheorem{theorem}{Theorem}
\newtheorem{lemma}{Lemma}
\newtheorem{puzzle}{Puzzle}
\newtheorem{corollary}{Corollary}
\newtheorem{proposition}{Proposition}


\begin{document}
\maketitle

\section{Administrative drivel}
\begin{itemize}
	\item Exam on friday!
		\begin{itemize}
			\item covers through today's lecture
			\item lecture images are up
			\item study guide has more in it than the exam will cover, but will be updated today to contain just the relevant info
		\end{itemize}
\end{itemize}

\section{Diseases}
\begin{itemize}
	\item clicker q: What would happen if a virus infected and disables your APCs? T-cells would no longer be able to recognize antigens
		\begin{itemize}
			\item This is not a common occurance!
			\item This is one of the major problems with HIV
			\item has anyone done this through Gain of function?
		\end{itemize}
	\item Infectious disease is:
		\begin{itemize}
			\item Transimittable
			\item communicable
			\item contagious
			\item you can catch it from someone else
		\end{itemize}
	\item Non-infectious diseases don't have the above properties
		\begin{itemize}
			\item E.g. lung cancer can't be transmitted to others (aside from through second hand smoke I suppose)
		\end{itemize}
	\item COVID is highly infectious
	\item the most important factor in spreading rate is the number of susceptible infectees
	\item probably know which diseases are infectious and not infectious out of the study guide list
	\item Types of infectioous diseases:
		\begin{itemize}
			\item Viruses
			\item bacteria
			\item fungi
			\item micro-organisms
			\item parasites
		\end{itemize}
	\item Example non-infectious diseases:
		\begin{itemize}
			\item Cancer (usually)
			\item kidney disease
			\item liver cirrhosis
			\item dietary difficiencies
			\item genetic disorders
		\end{itemize}
	\item Viruses:
		\begin{itemize}
			\item parts:
				\begin{itemize}
					\item spikes
					\item envelope
					\item capsid
					\item nucleic acid (RNA or DNA)
					\item some have head, sheath, tail fibers
					\item all but nucleic acids are made of protiens
				\end{itemize}
			\item they can only reproduce by taking over the transcription mechanism of a cell
			\item some viruses (bacteriaphages) kill bacteria
			\item no metabolism
			\item Non-living organisms
			\item NOT made of cells
				\begin{itemize}
					\item just a protein coat with DNA or RNA inside
				\end{itemize}
			\item DO NOT acquire and use energy
			\item CAN NOT reproduce without a host cell
			\item there are other non-living things that cause disease
				\begin{itemize}
					\item e.g. mad cow disease is just a self replicating protien
				\end{itemize}
		\end{itemize}
	\item Steps of viral infection:
		\begin{itemize}
			\item The surface has protiens that allow it to bind to cells
				\begin{itemize}
					\item they must be the right shape for the host cell
					\item so, who the virus attacks is fairly specific
				\end{itemize}
			\item They trick the cell into bringing the virus in
			\item Take over the host cell machinery to produce the protiens that their nucleic acids code for
				\begin{itemize}
					\item Reverse transcription is hijacked if RNA to turn it into DNA, transcription is hijacted if DNA
					\item makes copies of itself
				\end{itemize}
			\item Viral copies break off of the host, allowing them to spread
		\end{itemize}
	\item Viral diseases:
		\begin{itemize}
			\item Rabies
				\begin{itemize}
					\item no good treatments, deadly
				\end{itemize}
			\item flu
			\item cold
			\item measles
				\begin{itemize}
					\item You've been vaccinated for, yay
				\end{itemize}
			\item polio
				\begin{itemize}
					\item Jonas Salk gave us a vaccine in 1950s
					\item muscle wasting illness (yikes)
				\end{itemize}
			\item zika
				\begin{itemize}
					\item newer!
					\item suddenly became common, and made it to florida
				\end{itemize}
			\item HIV
				\begin{itemize}
					\item Taken out millions WW, 100ks in US in 80s, 90s
				\end{itemize}
			\item smallpox
				\begin{itemize}
					\item Edward Jenner did the thing to vaccinate, but we have a real one now.
					\item Only disease that we've eliminated from the wild! :D
						\begin{itemize}
							\item killed out by vaccinating almost everyone
							\item trying to do the same with polio, but wars are making it hard, along with cultures with stigmas against vaccines or outsiders
						\end{itemize}
					\item 2 places in the world have the virus in test tubes
				\end{itemize}
			\item hepatitis
				\begin{itemize}
					\item A,B,C,E, C most common, degenerates liver and can kill you
				\end{itemize}
			\item Ebola
		\end{itemize}
	\item Ebola -- scary -- symptoms
		\begin{itemize}
			\item doesn't tend to reside in humans (but wild life can transmit to humans)
			\item sudden fever
			\item intense weakness
			\item muscle pain and headache
			\item vomiting
			\item diarrhea
			\item impaired kidney and liver function
			\item internal and external bleeding
			\item usually results in death in a matter of days
			\item doesn't spread easily between humans, but is more common in cultures where people are in close contact with the dead bodies of infected
			\item HIGH  mortality -- about 50\% by as high as 90
			\item 2014-16 outbreak
				\begin{itemize}
					\item began in W. Africa
					\item more than 28,000 cases (+4 in US)
					\item more than 11k deaths (+1 in US)
				\end{itemize}
			\item Low infectiousness
				\begin{itemize}
					\item ONLY via direct contact with infected body fluids (of which there can be a lot)
				\end{itemize}
			\item Tracking epidemics (who.int has system of reporting cases)
			\item Subsequent smaller scale outbreaks are still happening
		\end{itemize}
	\item Influenza and colds
		\begin{itemize}
			\item Highly contagious
				\begin{itemize}
					\item person-to-person
					\item airborne
				\end{itemize}
			\item Flu has low martality rate
				\begin{itemize}
					\item usually less than 5 in 100k
					\item up to 100 in 100k during pandemics
					\item but high infectiousness == 20-30k+ deaths per year in the US, 291K-646K worldwide
						\begin{itemize}
							\item Almost all of these are elderly, the immune-compromised, or the residents of the poorest countries
						\end{itemize}
				\end{itemize}
			\item the flu will be with us forever...
			\item ``Common Cold" is caused by several different viruses
		\end{itemize}
	\item Seasonal influenza
		\begin{itemize}
			\item severall types
			\item Type A:
				\begin{itemize}
					\item H1N1 (1918 spanish flu, 2009 swine flu)
				\end{itemize}
			\item Type B:
			\item Type C:
		\end{itemize}
\end{itemize}

\end{document}
