\documentclass{article}
\usepackage[utf8]{inputenc}

\title{Week 5 Lecture 1}
\author{Jared Brannan }

\usepackage{natbib}
\usepackage{graphicx}
\usepackage{mathtools}
\usepackage{amsthm}
\usepackage{amsmath}
\usepackage{amssymb}
\usepackage{xcolor}
\usepackage{bbm}
\usepackage{bm}
\usepackage{physics}

% indent first line
\usepackage{indentfirst}
% one inch margins
\usepackage[margin=1.0in]{geometry}

\theoremstyle{definition}

\newcommand{\upRiemannint}[2]{
\overline{\int_{#1}^{#2}}
}
\newcommand{\loRiemannint}[2]{
\underline{\int_{#1}^{#2}}
}

\newtheorem{definition}{Definition}
\newtheorem{asside}{Asside}
\newtheorem{conjecture}{Conjecture}
\newtheorem{example}{Example}
\newtheorem{theorem}{Theorem}
\newtheorem{lemma}{Lemma}
\newtheorem{puzzle}{Puzzle}
\newtheorem{corollary}{Corollary}
\newtheorem{proposition}{Proposition}


\begin{document}
\maketitle

\section{Administrative drivel}
\begin{itemize}
	\item A few exams still need to be collected
	\item The first draft (polished!) is due Friday at Noon
\end{itemize}

\section{More on Cells}
\begin{itemize}
	\item Parts of cells:
		\begin{itemize}
			\item Membrane -- seperates in from out
			\item Cytoplasm -- liquid inside the ceell
				\begin{itemize}
					\item bulk of the volume
				\end{itemize}
			\item Organelles -- membrane-bound structures inside the cell
				\begin{itemize}
					\item Mitochondria (all eukariotes have) -- double membrane bound
						\begin{itemize}
							\item thought to previously have been independent organisms
							\item traded some of its genetic material with the genetic material of the host
							\item reproduce independently of the cell
							\item called the endo-symbiot hypothesis
							\item Mitochondria are calleld the "powerhouse" of the cell
						\end{itemize}
					\item Chloroplasts just in plant cells for photosynthesis -- double membrane bound
						\begin{itemize}
							\item Same sub items as mitochondria above, sans powerhouse
						\end{itemize}
					\item there are other organelles that are similar to these
					\item organelles do work for the cell
				\end{itemize}
		\end{itemize}
	\item clicker q: What would the best transport mechanism be for moving a charged ions accross a membrane from high to low concentration? Facilitated diffusion. (not passive diffusion, since ions can't cross the membrane on their own)
\end{itemize}
\subsection{Some organelles}
\begin{itemize}
	\item Nucleus
		\begin{itemize}
			\item only in eukareotic cells
			\item a prominant structure in the cell, one of the larger
			\item primary function: house the DNA
			\item parts: Nucleolus, Chromatin, Nuclear envelope, nuclear pore
			\item membrane bound, has 2 membranes
			\item surface covered in openings called nuclear pores that allow bigger things through
				\begin{itemize}
					\item allows mRNA to get out
					\item many other things can get through, but are beyond this course
				\end{itemize}
			\item Transciption takes place inside, producing mRNA
			\item  Nucleolus is involved in managing the chromosomes and DNA (unwinding, rewinding, transcription things, blah)
			\item Endoplasmic reticulum is attached to the outside (where the protiens are built)
			\item Chromotin:
				\begin{itemize}
					\item Is the DNA
					\item is an unwound chromosome
					\item DNA is wrapped around histones (proteins that give structure and organization to the chromosome)
					\item these bundles around the histones are called nucleosomes
					\item these are further wrapped into chromosomes
					\item ths structure is so small that enzymes and other molecules cant get to the information to make mRNA without unwinding the DNA.
					\item this is outside the nucleolus
					\item DNA structure:
						\begin{itemize}
							\item DNA is a double-helix, but the two helices are separate molecules. These two helices are held together with hydrogen bonds
							\item one helice is used to make an mRNA during transcription, and the other is used for error correction.
						\end{itemize}
				\end{itemize}
			\item First step in synthesizing a protein:
				\begin{itemize}
					\item A section of DNA (i.e. a gene) is copied into a strand of mRNA
						\begin{itemize}
							\item DNA unzipped
							\item Commplementary copy made 
							\item DNA rezips
						\end{itemize}
				\end{itemize}
		\end{itemize}
	\item Endopllasmic reticulum (ER)
		\begin{itemize}
			\item This is the membrane system close to and connected to the nucleus
			\item mRNA leaves the nucleus through a pore and into the ER
			\item is continuous with the membrane of the nucleus
			\item there is a rough and smooth ER
				\begin{itemize}
					\item the rough has ribosomes on it between layers
						\begin{itemize}
							\item ribosomes run the translation to build the polypeptites that will fold into proteins
							\item mRNA is read from one end to the other fand assembles the amino acid chain. (there's a start code and a stop code)
							\item reads a codon, grabs the amino acid and adds to the chain, then to the next codon, adds the amino acid, and so on.
							\item there are thousands, and each mRNA goes to exactly one ribosome
							\item made up of 80-90 protiens
							\item the polypeptide is released in the inner layers of the endoplasmic reticulum, where it's transported to an end of the ER, where a little vesel is formed to be transported to the gulgi apperatus
						\end{itemize}
					\item rough is directly connected to the nucleus and the smooth is connected to the rough
					\item smooth doesn't have any ribosomes, hence it is smooth
						\begin{itemize}
							\item doesn't recieve direct instruction from the nucleus
							\item produces fatty acids and steroids
							\item these are coded for indirectly (no genes)
							\item proteins that have been made in the rough ER come back from the gulgi aperatus to build fatty acids and steroids.
						\end{itemize}
				\end{itemize}
			\item hollow, with an internal space, and a space between layers
			\item 2 functions: smooth for lipids, rough for protiens
				\begin{itemize}
					\item the resultant molecules are retained or exocytoesd for use elsewhere
					\item protiens go to the Golgi aperatus where it's modified to become a functional protien (e.g. ends might be cut off)
					\item lipids are modified in the Golgi aperatus as well
				\end{itemize}
		\end{itemize}
	\item clicker Q: What do the blobs (green ribosomes) do? Translate mRNA sequence into a chain of amino acids
\end{itemize}


\end{document}
