\documentclass{article}
\usepackage[utf8]{inputenc}

\title{Week 13 Lecture 1}
\author{Jared Brannan }

\usepackage{natbib}
\usepackage{graphicx}
\graphicspath{ {./figures/} }
\usepackage{mathtools}
\usepackage{amsthm}
\usepackage{amsmath}
\usepackage{amssymb}
\usepackage{xcolor}
\usepackage{bbm}
\usepackage{bm}
\usepackage{physics}

% indent first line
\usepackage{indentfirst}
% one inch margins
\usepackage[margin=1.0in]{geometry}

\theoremstyle{definition}

\newcommand{\upRiemannint}[2]{
\overline{\int_{#1}^{#2}}
}
\newcommand{\loRiemannint}[2]{
\underline{\int_{#1}^{#2}}
}

\newtheorem{definition}{Definition}
\newtheorem{asside}{Asside}
\newtheorem{conjecture}{Conjecture}
\newtheorem{example}{Example}
\newtheorem{theorem}{Theorem}
\newtheorem{lemma}{Lemma}
\newtheorem{puzzle}{Puzzle}
\newtheorem{corollary}{Corollary}
\newtheorem{proposition}{Proposition}


\begin{document}
\maketitle

\section{Administrative drivel}
\begin{itemize}
	\item Scantrons have been processed, exam scores to be posted in the next few days
	\item term papers won't be looked at till next week
\end{itemize}

\section{Nutrition}
\begin{itemize}
	\item Malnutrition: a bigger problem for children
	\item Micronutrients
		\begin{itemize}
			\item Needed inn small amounts
				\begin{itemize}
					\item Not broken down, but used intact
					\item used individually, or used to modify macronutrient-based molecules
					\item may be part of a larger molecule, but the actual nutrient is not broken down
					\item used along with other molecules to do things
				\end{itemize}
			\item Vitamins
				\begin{itemize}
					\item Organic molecules -- they have a carbon backbone (carbon based)
					\item Often used in enzyme pathways, especially for metabolizing or synthesizing other organic molecules
						\begin{itemize}
							\item using them in the enzyme pathways makes the enzyme better at their job
						\end{itemize}
					\item only the organic molecules that we cannot generate on our own
						\begin{itemize}
							\item cows can produce almost all of the vitamins they need!
							\item this is done by gut bacteria
						\end{itemize}
					\item e.g. vitamin c
						\begin{itemize}
							\item involved in the productionn of colagen
							\item so, if you're vitamin c dificicient, you'll fall apart -- scurvey
							\item captain cook discovered that sour crout prevented the developement of scurvey, not knowing about vitamin C
						\end{itemize}
				\end{itemize}
			\item minerals
				\begin{itemize}
					\item Inorganic molecules -- non-carbon based
					\item often used as ions or crystals
						\begin{itemize}
							\item e.g. salt is split into 2 ions 
						\end{itemize}
				\end{itemize}
			\item these cannot be constructed by the body, and need to be injested
		\end{itemize}
	\item List of vitamins:
		\begin{itemize}
			\item Fat-soluble: A, D, E, K
				\begin{itemize}
					\item very common
					\item disolve in fat
						\begin{itemize}
							\item doessn't dissolve well in water
							\item obsorbs into lipids, fatty tissues
							\item so, it's hard to get rid of excess, so it is stored as deposits in fatty tissues, which can become toxic
						\end{itemize}
					\item A important for light detection
					\item D is important in growth; bones, muscles
					\item E important in cell membranes, esp neuro cells
					\item K important for reasons?
				\end{itemize}
			\item Water-soluble: B (1 (thiamine), 2 (riboflavin), 3 (niacin), 5, 6, 7 (biotin), 9 (folic acid), 12), C
				\begin{itemize}
					\item folic acid is important during fetal developent, particularly in the spine; not great for males to consume
					\item disolves in water, so excess is dumped from the kidneys into the urine, so it's virtually impossible to have too much
				\end{itemize}
			\item modern food is fortified with vitamins
		\end{itemize}
	\item micro -- water soluble vitamins
		\begin{itemize}
			\item small quantity needed
			\item very difficult to overdose
			\item excess is excreted in urine
		\end{itemize}
	\item Vitamin C
		\begin{itemize}
			\item Aids in the synthesis of collagen -- important for wound healing
			\item scurvy: fatigue, bleeding from mucous membranes, spongy gums, tooth loss, open wounds
				\begin{itemize}
					\item caused by vitamin C difficiency
					\item ultimately leads to death
					\item common in sailors of the past
				\end{itemize}
		\end{itemize}
	\item fat-soluble vitamins
		\begin{itemize}
			\item needed in small quantity
			\item overconsumption can lead to toxicity
			\item not easy excreted in urine, so are stored in the liver
				\begin{itemize}
					\item Hamper liver function when in excess
				\end{itemize}
		\end{itemize}
	\item Vitamin A
		\begin{itemize}
			\item beta-carotene -- the pigment that causes the color, and vitamin A is derived from it
				\begin{itemize}
					\item found in meat, oorange fruits and veggies
					\item Immune functioning
					\item DNA $\to$ RNA (transcription)
					\item Vission
				\end{itemize}
			\item Royal Airforce (british) 1939 -- Airborne Interception Radar
				\begin{itemize}
					\item advertised that airmen needed vitamin A in order to see at night
					\item early WWII
				\end{itemize}
		\end{itemize}
	\item List of minerals;
		\begin{itemize}
			\item Calcium (Ca)
				\begin{itemize}
					\item important in bones and muscle function
				\end{itemize}
			\item Phosphorus (P)
			\item Potasium (K)
			\item Solfur (S)
			\item Sodium (NA)
			\item Chlorine (Cl)
			\item Magnesium (Mg)
			\item Iron (Fe)
				\begin{itemize}
					\item important in blood: hemoglobin
				\end{itemize}
			\item Iodine (I)
			\item Manganese (Mn)
			\item Copper (Cu)
			\item Cobalt (Co)
			\item Zinc (Zn)
			\item Fluorine (Fl)
				\begin{itemize}
					\item prevents cavities, since it's highly reactive
				\end{itemize}
			\item Selenium (Se)
			\item Chromium (Cr)
		\end{itemize}
	\item minerals:
		\begin{itemize}
			\item Some needed in larger quantities (Na, Cl, ,Ca, P)
			\item Others in trace amounts (S, Co, Mn, Se)
		\end{itemize}
	\item Recommended diet
		\begin{itemize}
			\item Macronutrients are easy to come by in most foods
				\begin{itemize}
					\item most difficult to get is probably protiens -- getting all of the amino acids
				\end{itemize}
			\item conusuming too much is the typical problem
			\item micronutrients can be more challenging
			\item Not too much 
				\begin{itemize}
					\item eat enough calories but not more than you use
				\end{itemize}
			\item High diversity
				\begin{itemize}
					\item anything in excessive quantity can be bad for you
				\end{itemize}
			\item Lots of plants
				\begin{itemize}
					\item relatively high in micronutrients
					\item relatively low in macronutrients
					\item macroos that are present are usually more complex == more energetically expensive to digest
				\end{itemize}
		\end{itemize}
	\item clicker Q: Why are vitamins and minerals called micro-nutrients? Your body needs these in small amounts
	\item end of nutritian
\end{itemize}

\section{Digestive system}
\begin{itemize}
	\item clicker q: what hormone is released by the pancreas when blood sugar is high? Insulin
	\item Functions:
		\begin{itemize}
			\item Break down food
			\item absob nutrients
			\item dispose of waste
		\end{itemize}
	\item Tow major divisions:
		\begin{itemize}
			\item \textbf{Gastrointestinal (GI) tract}:
				\begin{itemize}
					\item connects mouth $\to$ stomack $\to$ intestines (small and large) $\to$ anus
				\end{itemize}
			\item \textbf{Accessory organs} 
				\begin{itemize}
					\item Salivary glands
						\begin{itemize}
							\item food lubrication
						\end{itemize}
					\item pancreas
						\begin{itemize}
							\item produces lots of hormones
						\end{itemize}
					\item liver
						\begin{itemize}
							\item detoxifies food
						\end{itemize}
					\item gallbladder
					\begin{itemize}
						\item breaks down lipids
					\end{itemize}
				\end{itemize}
		\end{itemize}
\end{itemize}


\end{document}
