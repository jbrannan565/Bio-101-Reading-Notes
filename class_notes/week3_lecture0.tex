\documentclass{article}
\usepackage[utf8]{inputenc}

\title{Week 3 Lecture 0}
\author{Jared Brannan }

\usepackage{natbib}
\usepackage{graphicx}
\usepackage{mathtools}
\usepackage{amsthm}
\usepackage{amsmath}
\usepackage{amssymb}
\usepackage{xcolor}
\usepackage{bbm}
\usepackage{bm}
\usepackage{physics}

% indent first line
\usepackage{indentfirst}

\theoremstyle{definition}

\newcommand{\upRiemannint}[2]{
\overline{\int_{#1}^{#2}}
}
\newcommand{\loRiemannint}[2]{
\underline{\int_{#1}^{#2}}
}

\newtheorem{definition}{Definition}
\newtheorem{asside}{Asside}
\newtheorem{conjecture}{Conjecture}
\newtheorem{example}{Example}
\newtheorem{theorem}{Theorem}
\newtheorem{lemma}{Lemma}
\newtheorem{puzzle}{Puzzle}
\newtheorem{corollary}{Corollary}
\newtheorem{proposition}{Proposition}


\begin{document}
\maketitle

\section{Administrative drivel}
\begin{itemize}
	\item Exam 1 -- Friday Spt. 17 

		\begin{itemize}
			\item in class on paper up through the first biochistry lecture (carbs/lipids)
			\item about 40 questions
			\item focus on class note content
		\end{itemize}
\end{itemize}

\section{More on human evolution}
\begin{itemize}
	\item Major glacial events 15k years ago allowed for movement across large bodies of water due to sea level drop
	\item The rockies were one of the only places that weren't ice in the americas, so movement went along there
\end{itemize}
\subsection{brain size}
\begin{itemize}
	\item Some/all of this was due to our changes in behavior,
	\item a lot of our domination was due to increased brain size, and it's structure (probably more to do with structure)
	\item Human astrocytes (brain cells responsible for human brain connections) were injected into mouse brains and they were much better at problem solving as a result.
	\item in fossiles, we can deetermine a lot about brain size and structure by surface patterns on the skull.
	\item problem solving abilities allows humans to live in a large variety of enviornments
	\item Homo erectus was alive after Neanderthal, and so was the last other member of human's genus to be alive, and died out ~10k years ago.
\end{itemize}

Note on filogenetic trees: if a branch doesn't make it to the far right, it went extinct.

\subsection{Video on human evolutionary history}
\begin{itemize}
	\item Keurgesagt -- What happened before history? Human origins
\end{itemize}

\section{Biochemistry: The Chemistry of Life}
Living things "contain materials found only in living organisms"

\subsection{How do substances interatct with each other?}
\begin{itemize}
	\item Atoms are the small things that interact with each other and make up all of atoms
		\begin{itemize}
			\item The fundamental unit of matter
			\item every atom is of a particular element (carbon, gold, etc.)
			\item atoms can be converted into other elements, but it's hard
			\item There are a lot of elements, but life depends on a handfull of the smaller of them.
			\item periodic table is organized based on their basic structure (mostly their electrongs), which determine how they interact.
			\item each column will react with the other columns in similiar ways
		\end{itemize}
	\item chemicals are composed of attoms
	\item All things are made of chemicals
\end{itemize}

\subsection{What elements are you made of}
\begin{itemize}
	\item There are about 120 elements in humans
	\item rely mostly on oxygen, carbon, hydrogen, nitrogen -- top 4 in human mass (96\%)
	\item going down the rows in periodic table, the mass increases
	\item other important ones: calcium, phosphorus in bones, sodium, clorine in salts.
\end{itemize}

\subsection{Parts of the atom}
\begin{itemize}
	\item Electrons, Neutrons, Protons, nucleus 
		\begin{itemize}
			\item nuclieus: the ball of neutrons and protons
			\item nutrons/protons have mass atomic unit of 1
			\item protons have +1 charge
			\item electrons have -1 charge
			\item neutrons have neutral charge
			\item almost all of an atom is empty space
		\end{itemize}
	\item the number of protons define what the element \textit{is}
		\begin{itemize}
			\item hydrogen has 1 proton
			\item helium as 2
			\item oxygen has 8
			\item carbon has 6
		\end{itemize}
	\item the number of neutrons can vary, and is ussually about the same as the number of protons
	\item electrons drive the interactions between attoms, and are very small (essentially no size or mass)
\end{itemize}

More than half of the human body mass is made up of \textit{Carbon and Oxygen}.


\end{document}
