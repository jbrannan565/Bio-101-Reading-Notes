\documentclass{article}
\usepackage[utf8]{inputenc}

\title{Week 6 Lecture 0}
\author{Jared Brannan }

\usepackage{natbib}
\usepackage{graphicx}
\usepackage{mathtools}
\usepackage{amsthm}
\usepackage{amsmath}
\usepackage{amssymb}
\usepackage{xcolor}
\usepackage{bbm}
\usepackage{bm}
\usepackage{physics}

% indent first line
\usepackage{indentfirst}
% one inch margins
\usepackage[margin=1.0in]{geometry}

\theoremstyle{definition}

\newcommand{\upRiemannint}[2]{
\overline{\int_{#1}^{#2}}
}
\newcommand{\loRiemannint}[2]{
\underline{\int_{#1}^{#2}}
}

\newtheorem{definition}{Definition}
\newtheorem{asside}{Asside}
\newtheorem{conjecture}{Conjecture}
\newtheorem{example}{Example}
\newtheorem{theorem}{Theorem}
\newtheorem{lemma}{Lemma}
\newtheorem{puzzle}{Puzzle}
\newtheorem{corollary}{Corollary}
\newtheorem{proposition}{Proposition}


\begin{document}
\maketitle

\section{Administrative drivel}
\begin{itemize}
	\item
\end{itemize}

\section{More on tissues...}
\begin{itemize}
	\item \textbf{Connective tissue/Extracellular matrix} 
		\begin{itemize}
			\item Usually connect things together
			\item anything outside/in-between cells/organs -- cels embedded in a non-celluar matrix
			\item connection/integration, protection, and storage
			\item Connectivity
				\begin{itemize}
					\item e.g. tendons connect muscle to bone
				\end{itemize}
			\item Strength
				\begin{itemize}
					\item e.g. the support of muscle tissue
				\end{itemize}
			\item Protection
				\begin{itemize}
					\item e.g. bones protect tissue.
				\end{itemize}
			\item cells in connective tissue destroy einvading microrganisms (white blood cells)
				\begin{itemize}
					\item Note: blood is a connective tissue! Blood cells are suspended in plasma, a non-cellular matrix!
				\end{itemize}
			\item common molecules in extra-cellular matrix: collagen, elastin
			\item Ct/Ecm
				\begin{itemize}
					\item Cells secrete proteins and carbohydrates that form the extracellular matrix
					\item Structural and connection/communication between cells
					\item \textbf{Bone} : Collagen proteins + minerals + [bone cells]
					\item \textbf{Skin}: (beneath the cells) elastin and collagen proteins + various cells of the skin
					\item Technically that blood and lymph system also are connective/extracellular tissues
				\end{itemize}
			\item clicker Q: The epithelial cells contribute to defense against infection, but \textbf{not} by destroying invading microorganisms
		\end{itemize}
	\item Muscle tissue -- contractile
		\begin{itemize}
			\item 3 subtypes
			\item \textbf{skeletal} type -- moving limbs, etc -- activate by nerve cells
				\begin{itemize}
					\item we will meet these cells in more detail later
					\item go from a long to a shorter condition (contraction), cannot push
					\item this type is strong and generates large forces
					\item fatigues quickly -- rest interval required between use
				\end{itemize}
			\item \textbf{Cardiac} type -- the heart has a specific type of muscle cell 
				\begin{itemize}
					\item they contract without the need of an external signal
					\item only in the heart
					\item have a different shape than the other 2 types
					\item produces a modest amount of force to circulate blood
						\begin{itemize}
							\item beats about once a second until you die
							\item this requires tremendous endurance
						\end{itemize}
				\end{itemize}
			\item \textbf{Smooth}  type -- layers in intestines, urinary systeme, reproductive organs
				\begin{itemize}
					\item changes the length and diameter of tubes
					\item one of the largest muscles in the body is the uterus!
					\item has moderate endurance and produces more modest forces, and needs a rest interval
				\end{itemize}
		\end{itemize}
	\item Nervous tissue -- excitable cells and supporting cells
		\begin{itemize}
			\item sensing the environment
				\begin{itemize}
					\item internal and external
						\begin{itemize}
							\item internal is unconsious, this prevents us from going ``nuts"
						\end{itemize}
				\end{itemize}
			\item organizing the appropriate response
				\begin{itemize}
					\item e.g. messages to leg muscles
				\end{itemize}
			\item made up of nerve cells that change their chemical composition to sent messages
			\item Also keeps things ticking along (e.g. movement in the intestines)
			\item we'll encounter these again later.
		\end{itemize}
	\item tissues: multiple cells of the same type
	\item organ: multiple tissue types together to fulfill some function
		\begin{itemize}
			\item Muscle + touch elastic coat + slippery inner lining + nerves = HEART :)
		\end{itemize}
	\item organs make up organ systems!
		\begin{itemize}
			\item e.g. Heart + arteries + veins + capillaries = Cardiovascular system!
				\begin{itemize}
					\item arteries carry blood from the heart to parts of the body
					\item veins carry blood from the extremities to the heart
					\item capillaries carry blood throughout the tissues, carrying blood from the arteries to the veins.
						\begin{itemize}
							\item all of the cells needed molecules are carried to them through the capillaries!
						\end{itemize}
				\end{itemize}
			\item others: skeletomuscular, nervous, integumentary, lymphatic and immune, circulatory, repiratory, digestive, urinary, endocrine, reproductive
				\begin{itemize}
					\item (might not cover the last few of these)
				\end{itemize}
			\item example: skin has all of the tissue types!
		\end{itemize}
	\item all of the organ systems together make up an organism!
\end{itemize}

\section{Organs}



\end{document}
