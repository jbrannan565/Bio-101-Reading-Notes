\documentclass{article}
\usepackage[utf8]{inputenc}

\title{Week 9 Lecture 2}
\author{Jared Brannan }

\usepackage{natbib}
\usepackage{graphicx}
\graphicspath{ {./figures/} }
\usepackage{mathtools}
\usepackage{amsthm}
\usepackage{amsmath}
\usepackage{amssymb}
\usepackage{xcolor}
\usepackage{bbm}
\usepackage{bm}
\usepackage{physics}

% indent first line
\usepackage{indentfirst}
% one inch margins
\usepackage[margin=1.0in]{geometry}

\theoremstyle{definition}

\newcommand{\upRiemannint}[2]{
\overline{\int_{#1}^{#2}}
}
\newcommand{\loRiemannint}[2]{
\underline{\int_{#1}^{#2}}
}

\newtheorem{definition}{Definition}
\newtheorem{asside}{Asside}
\newtheorem{conjecture}{Conjecture}
\newtheorem{example}{Example}
\newtheorem{theorem}{Theorem}
\newtheorem{lemma}{Lemma}
\newtheorem{puzzle}{Puzzle}
\newtheorem{corollary}{Corollary}
\newtheorem{proposition}{Proposition}


\begin{document}
\maketitle

\section{Administrative drivel}
\begin{itemize}
	\item exams will be by monday 
\end{itemize}

\section{Anatomy and Physiology}
\subsection{Respiratory system}
\begin{itemize}
	\item More on asthma:
		\begin{itemize}
			\item Immune response when the bronchial lining are irritated
				\begin{itemize}
					\item Inflamation response (covered in immune system)
					\item swelling, fluids, mucuus into the lumen, to try to catch the offenders
				\end{itemize}
		\end{itemize}
	\item Other diseases -- infections -- pneumonia (bacterial, viral) and tuberculosis (bacterial) infection
		\begin{itemize}
			\item tuberculosis is super deadly
				\begin{itemize}
					\item once called ``consumption"
					\item common in fairly large cities
					\item spread by cough droplets
					\item The old method of treatment was to quarenteen with really fresh air, at a sanatorium for months to a year, where you either died or recovered
					\item WWII first treatments were made with the invention of penisilin!
					\item illness kills a lot of soldiers, so having penisilin was a big advantage
					\item there  are now strains of tb that are antibiotic resistant
				\end{itemize}
			\item the infectees make their living by extracting nutrients from you, which damages your tissues!
			\item Starts with a small population which grows
			\item  Alviolis swells due to an immune response
			\item pneumonia
				\begin{itemize}
					\item much more common in older people
				\end{itemize}
			\item these should be treated as an ecological process
				\begin{itemize}
					\item looking at the growth of the population as exponential
				\end{itemize}
		\end{itemize}
	\item \textbf{Emphysema} 
		\begin{itemize}
			\item Characterized by alveolar walls rupture to form larger sacs and build up of environmental pollutants
			\item when the alveolus burst, they fuse with a neighbor making these sacks
			\item this decreases the surface area of the lungs
			\item very common in smokers and coal miners
			\item in mid-late stage, the patient needs an oxygen tank
			\item can eventually kill you
		\end{itemize}
	\item \textbf{Lung cancer} 
		\begin{itemize}
			\item very serious
			\item treatments are much better than they were
				\begin{itemize}
					\item if you got it in the 70s you were doomed
				\end{itemize}
			\item chemical treatments destroy new fast deviding cells in tumors, but also damaging older tissues
			\item radiation is used
			\item newer treatements deploy an imune response against cancer tissues (largeley experimental)
			\item increased risk from iritants
		\end{itemize}
	\item \textbf{Pulmanary fibrosis} 
		\begin{itemize}
			\item Environmental irritants, asbestos, autoimmune causes. Lung tissues are replaced with scar tissue.
		\end{itemize}
	\item clicker q: by what mechanism does $CO_2$ move from the blood into the lungs? diffusion.
	\item end of respiritory system
\end{itemize}


\end{document}
