\documentclass{article}
\usepackage[utf8]{inputenc}

\title{Week 2 Lecture 1}
\author{Jared Brannan }

\usepackage{natbib}
\usepackage{graphicx}
\usepackage{mathtools}
\usepackage{amsthm}
\usepackage{amsmath}
\usepackage{amssymb}
\usepackage{xcolor}
\usepackage{bbm}
\usepackage{bm}
\usepackage{physics}

% indent first line
\usepackage{indentfirst}

\theoremstyle{definition}

\newcommand{\upRiemannint}[2]{
\overline{\int_{#1}^{#2}}
}
\newcommand{\loRiemannint}[2]{
\underline{\int_{#1}^{#2}}
}

\newtheorem{definition}{Definition}
\newtheorem{asside}{Asside}
\newtheorem{conjecture}{Conjecture}
\newtheorem{example}{Example}
\newtheorem{theorem}{Theorem}
\newtheorem{lemma}{Lemma}
\newtheorem{puzzle}{Puzzle}
\newtheorem{corollary}{Corollary}
\newtheorem{proposition}{Proposition}


\begin{document}
\maketitle
\section{Administrative drivel}
\begin{itemize}
	\item First assignment is up on canvas, due tomorrow
	\item Clickers are still down, so we're doing in class attendence.
\end{itemize}

\section{Continuing human evolutionary history}

Beginning with some review...
\subsection{~20 million years ago}
\begin{itemize}
	\item Great apes evolve in the african savana
	\item note: the ecological state is very different then it is in africa today. Grasses are pretty recent, and prier to that there were no grasslands, praries, or savanahs. Most places were woodlands and forests, and the conifers could tollerate less moisture than the broad leaf trees. It was only on the order of about ~100 million years ago that we first have andriosperms (flowering and fruit bearing plants). Before that there were cone bearing plants. The flowing plants are what we think of today that animals eat, while to dinosaurs ate mostly coning plant material. Mamals focused on eating flowering plants, which has allowed for the prolifferation of the kinds of mamals we see.
	\item Grasslands are open (very different from moist forests: different food, predators). In a forest the best route of escape is up a tree, which you can't do in a grassland. The best defense in grasslands was to be in a group and use tools to defend yourself. This need lead to primates becoming more social creatures.
	\item Grasses are flowering plants, and polinate mostly by the wind.
\end{itemize}

\subsection{~7-8 million years ago}
\begin{itemize}
	\item first common ansestor between chimps and humans is here
\end{itemize}

New stuff next
\subsection{~3.5 million years ago}
\begin{itemize}
	\item genus Australopithecus appears, with species
		\begin{itemize}
			\item afarensis
			\item anamensis
			\item africanus (big jaw)
			\item more
		\end{itemize}
	\item most famous specimine: lucy
		\begin{itemize}
			\item She was an Australopithecus afarensis
			\item She was named after the beatles song "Lucy in the Sky with Diamonds"
		\end{itemize}
	\item First bipedal primate!
		\begin{itemize}
			\item Lived in grasslands,  so better to see further
			\item Frees hands to use tools (stone)
				\begin{itemize}
					\item tools required a lot of knowledge  to make more advanced tools
					\item hand axes didn't have handles
				\end{itemize}
			\item Long arms, so probably still partially  arboreal
				\begin{itemize}
					\item arms to the side, rather than more underneath like in dogs or horses
					\item probably an artifact of predecessors living in trees
				\end{itemize}
		\end{itemize}
\end{itemize}

\subsection{~2.5 million years ago}
\begin{itemize}
	\item Homo habilis splits off from Australopithecus lineage
	\item consiterably larger brain, relative to body size
		\begin{itemize}
			\item Diet change from mostly herbivore to some meat
		\end{itemize}
	\item stone, and maybe bone tools
	\item almost everyone thought humans came from west africa, but "Leaky"s wife discovered homo habilis.
\end{itemize}

\subsection{~1.8 million years ago}
\begin{itemize}
	\item Homo ergaster and H. erectus
	\item First hominids dto expand into Europe and Asia from Africa
	\item Large populations and longer lives, so we know a lot
	\item direct predecesor to homo sapiens
	\item Control of fire for heat  and warmth, uncertain about cooking (first fire users!)
		\begin{itemize}
			\item Carried it around with them
			\item used it to defend against night predators
			\item especially important with earth's cooling, since there were multiple ice ages.
		\end{itemize}
	\item more-developed stone tools
		\begin{itemize}
			\item hand axes developed handles
			\item first spears with heads
			\item better spear heads and axe heads
		\end{itemize}
	\item Approximately human proportioned body (looked prety close to humans)
\end{itemize}

\subsection{~800k years ago}

\begin{itemize}
	\item Homo heidelbergensis arises from H. erectus
	\item lived in Africa, Europe W. Asia
	\item Still larger brain (almost as large as modern humans)
	\item Discovered within the last few decades
\end{itemize}

\subsection{~400k years ago}
\begin{itemize}
	\item Homo neanderthalensis, descendeent of H.heidel... in Eurasia (first not evolved in Africa)
	\item Large brain (up to \%20 larger than humans
		\begin{itemize}
			\item Exttensive use of tools, shelters, fire, even boats (not quite ocean fairing)
			\item probably clothing, at least hides as blankets and covers
		\end{itemize}
	\item Broad body allowed for cold tolerance
		\begin{itemize}
			\item But limited their hunting ability, making them less effecient at hunting than humans would be
			\item humans could easily outrun
			\item evolved in cold climate
			\item short limbs, smaller appendages in general (nose, ears, etc)
		\end{itemize}
\end{itemize}

\subsection{~200k years ago}
\begin{itemize}
	\item Homo sapiens also descended from H. Heidel..., but in Africa
	\item pretty dry at the time
	\item Lighter-weight skeleton
	\item Co-existed with Neanderthals for 20-30k years, with interbreeding (Neanderthals extinct  ~40k years ago)
		\begin{itemize}
			\item Hypotheses about this extinction: human interbreeding or out competing
		\end{itemize}
	\item Neanderthals likely had slightly larger eye volume
	\item Neanderthals likely had slightly smaller group sizes
	\item the same region of the brain that handle social interactions is also involved in higher-level cognition (interative comprehension; reasoning; mind; language)
	\item these regions are better developed in humans than in neanderthals
	\item skulls grow with the surface of the brain, so we can see the brain structure (loosley) from fossils of skulls. 
\end{itemize}



\end{document}

