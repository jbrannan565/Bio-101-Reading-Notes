\documentclass{article}
\usepackage[utf8]{inputenc}

\title{Week 7 Lecture 0}
\author{Jared Brannan }

\usepackage{natbib}
\usepackage{graphicx}
\usepackage{mathtools}
\usepackage{amsthm}
\usepackage{amsmath}
\usepackage{amssymb}
\usepackage{xcolor}
\usepackage{bbm}
\usepackage{bm}
\usepackage{physics}

% indent first line
\usepackage{indentfirst}
% one inch margins
\usepackage[margin=1.0in]{geometry}

\theoremstyle{definition}

\newcommand{\upRiemannint}[2]{
\overline{\int_{#1}^{#2}}
}
\newcommand{\loRiemannint}[2]{
\underline{\int_{#1}^{#2}}
}

\newtheorem{definition}{Definition}
\newtheorem{asside}{Asside}
\newtheorem{conjecture}{Conjecture}
\newtheorem{example}{Example}
\newtheorem{theorem}{Theorem}
\newtheorem{lemma}{Lemma}
\newtheorem{puzzle}{Puzzle}
\newtheorem{corollary}{Corollary}
\newtheorem{proposition}{Proposition}


\begin{document}
\maketitle

\section{Administrative drivel}
\begin{itemize}
	\item
\end{itemize}

\section{Anatomy and Physiology}

\subsection{Muscles}
\begin{itemize}
	\item The ends of the sarcomeres are reffered to sometimes as zlines
	\item Contraction
		\begin{itemize}
			\item Actin-myosin cycle:
				\begin{itemize}
					\item Myosin binds to actin
					\item myosin changes position, pulling on actin
					\item myosin releases actin, ATP binds at this final step
				\end{itemize}
			\item repeating this cycle shortens the sarcomere.
			\item If ATP is absent, myosin cannot release
			\item rigor mortis sest  in because after death ATP dwindles and eventually runs out
				\begin{itemize}
					\item myosin is left permanently bound to the actin
				\end{itemize}
			\item this is the same thing as a cramp!
			\item Nerve cells stimulate muscle contraction:
				\begin{itemize}
					\item \textit{Nerve}  cells release neurotransmitter (chem. signal)
					\item It \textit{Binds to receptors}  in the plasma membrane of the nuscle cell (fiber)
					\item Causes increases in calcium ions ($CA^{2+}$) inside the cell (stored in the smooth ER/ sarcoplasmic reticulum)
						\begin{itemize}
							\item each actin is normally covered by an accessory protien, but the ions open up the actin for binding
						\end{itemize}
					\item Calcium causes shift in protein (tropomyosin) that blocks myosin from binding to actin
					\item Myosin can interact with actin $\to$ contraction
				\end{itemize}
			\item Both $CA^{2+}$ and ATP are involved
		\end{itemize}
	\item Relaxation
		\begin{itemize}
			\item Calcium is pumped back into the SER
			\item w/o CA++, cacium regulated protein blocks myosin binding site on actin
			\item myosin cannoot bind any more / no more contraction
			\item Muscle relaxation occurs as antagonistic muscle dooes it's job via the tendons, pulls the muscle back to the relaxed length
		\end{itemize}
	\item To recap:
		\begin{itemize}
			\item Basic structure: sarcomere
			\item made of thin filament of actin, thick of myocin with movable heads
			\item nervous system sends a nerotransmitter, traveling to a T tubiul, down SER, leading to the relase of CA++
			\item the CA++ acts on the accessory protiens
				\begin{itemize}
					\item 2 kinds: troponin and tropoomyosin (former pulls on the latter to expose the binding sites on the actin when exposed to CA++)
				\end{itemize}
			\item At the sarcomere, myosin heads 'walk' along actin filament
				\begin{itemize}
					\item myosin binds to actin
					\item ATP is broken, releasing the phosphate, moving the myosin head to the next binding site
				\end{itemize}
			\item ~20 cm of contraction is spread across ~100,000 sarcomeres
		\end{itemize}
	\item Different types of muscle fiber: slow twitch and fast twitch
		\begin{itemize}
			\item Almost all muscles have both, and one might dominate the other depending on use and genes, etc
			\item slow twitch develops with arobic, fast twitch with quick motions
			\item slow twitch has high endurance relative to fast twitch
			\item slow twitch
				\begin{itemize}
					\item lots of blood supply
					\item lots of mitocondria
					\item lots of myoglobin (binds to oxygen, much like hemoglobin)
					\item less gglycogen
					\item more stamina
					\item less diameter
					\item less tension
					\item Aerobic -- requires high $O_2$ delivery
						\begin{itemize}
							\item $O_2$ present
							\item Glucose $\to$ pyruvic acid + 2ATP
							\item pyruvic acid $\to$ $CO_2 + H_2O$ + 34 ATP
						\end{itemize}
					\item most of what we do is slow twitch
					\item small movments, sitting upriight, endurance
				\end{itemize}
			\item fast twitch
				\begin{itemize}
					\item Lower blood supply
					\item fewer mitochondria
					\item Less myoglobin
					\item Lots of glycogen
					\item less stamina
					\item bigger diameter
					\item more tension
					\item anaerobic -- does not require high $O_2$
						\begin{itemize}
							\item $O_2$ absent
							\item Glucose $\to$ pyruvic acid + 2 ATP
							\item pyruvic acid $\to$ Lactic acid
								\begin{itemize}
									\item (lactic acid is eventually split into pyruvic acid and put back throuogh aerobic cycle)
								\end{itemize}
						\end{itemize}
					\item breif powerful movment
				\end{itemize}
		\end{itemize}
	\item \textbf{Hypertorphy}  is due to increase in number of myofibrils, but not the number of cells (think body builders)
		\begin{itemize}
			\item Each muscle cell has multiple myofibrils
		\end{itemize}
	\item \textbf{Atrophy}  loss of muscle mass is a normal part of aging, and is impacted by nutritian , exercise. Also occurs if a limb is inactive for several months
	\item Anabolic steroids:
		\begin{itemize}
			\item Testosterone-like chemicals, stimulates actin and myosin production
			\item cheating, long term side effects
		\end{itemize}
	\item Clicker Q: calcium does NOT stimulate ATP breakdown
	\item end of play...
\end{itemize}

\section{Cardiovascular system}
\subsection{Heart}
\begin{itemize}
	\item In the thoracic cavity and protected by the ribs
	\item has 4 pumps that push blood through the pulmonary and systemic circuits
	\item Cardo == heart
	\item vasular == blood vessels
	\item Functions of cardio system:
		\begin{itemize}
			\item blood picks up all of the waste from cells/tissues (e.g.$CO_2$)
			\item blood caries things to the cells/tissues (e.g. $O_2$, glucose)
			\item Transport immune cells
			\item causees clotting
			\item trasport hormones
			\item transports water
			\item etc.
		\end{itemize}
	\item What color is blood?
		\begin{itemize}
			\item books show blood as red inn the arteries, and blue in the veins
			\item blood is \textit{not} blue
			\item oxygenated blood is bright red, deoxygenated blood is a little darker red
		\end{itemize}
\end{itemize}




\end{document}
