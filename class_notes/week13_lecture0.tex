\documentclass{article}
\usepackage[utf8]{inputenc}

\title{Week 13 Lecture 0}
\author{Jared Brannan }

\usepackage{natbib}
\usepackage{graphicx}
\graphicspath{ {./figures/} }
\usepackage{mathtools}
\usepackage{amsthm}
\usepackage{amsmath}
\usepackage{amssymb}
\usepackage{xcolor}
\usepackage{bbm}
\usepackage{bm}
\usepackage{physics}

% indent first line
\usepackage{indentfirst}
% one inch margins
\usepackage[margin=1.0in]{geometry}

\theoremstyle{definition}

\newcommand{\upRiemannint}[2]{
\overline{\int_{#1}^{#2}}
}
\newcommand{\loRiemannint}[2]{
\underline{\int_{#1}^{#2}}
}

\newtheorem{definition}{Definition}
\newtheorem{asside}{Asside}
\newtheorem{conjecture}{Conjecture}
\newtheorem{example}{Example}
\newtheorem{theorem}{Theorem}
\newtheorem{lemma}{Lemma}
\newtheorem{puzzle}{Puzzle}
\newtheorem{corollary}{Corollary}
\newtheorem{proposition}{Proposition}


\begin{document}
\maketitle

\section{Administrative drivel}
\begin{itemize}
	\item Exam grades are not in yet
	\item we're throwing out the senses, since we won't be worried about this  till we're in our 50s
	\item So, we'll proceed to nutrition and digestion
	\item next week we'll cover ecosystem processes and human impact
\end{itemize}

\section{Nutrition}
\begin{itemize}
	\item Why do we eat?
		\begin{itemize}
			\item We must acquire from the enviornment the energy and building blocks to run our activities
			\item Building blocks
				\begin{itemize}
					\item Things that don't provide energy
					\item provide molecules for building structures, protiens, etc
				\end{itemize}
			\item calories (energy)
				\begin{itemize}
					\item Plants can turn light into chemical energy, but we have to get our energy through calories in food
					\item the principle form of chemical energy for humans is glucose
					\item The energy we get ultimately comes from plants, who get it from the sun
				\end{itemize}
		\end{itemize}
	\item Nutrients -- the things we need to survivie
		\begin{itemize}
			\item Macronutrients -- need lots
				\begin{itemize}
					\item carbohydrates
						\begin{itemize}
							\item bulk of the mass of plants
						\end{itemize}
					\item lipids
					\item protiens
						\begin{itemize}
							\item Heavy meat intake leads to weight gain, and cardiovascular disease with potential shortened life
						\end{itemize}
				\end{itemize}
			\item Micronutrients -- need small amounts
				\begin{itemize}
					\item vitamins
					\item minerals
				\end{itemize}
		\end{itemize}
	\item macronutrients:
		\begin{itemize}
			\item \textbf{Protiens}
				\begin{itemize}
					\item Amino acids, polypeptides
					\item Importantly, they get turned into digestive enzymes
					\item Recycle into your own proteins 
						\begin{itemize}
							\item break apart ingested proteins into individdual amino acids
							\item Assemble amino acids into proteins at the ribosomes
						\end{itemize}
					\item Some are built into nucleic acids
						\begin{itemize}
							\item recall that both hamino acids are nucleic acides are N-based (nitrogen based)
						\end{itemize}
					\item Minor source of energy
						\begin{itemize}
							\item turned into glucose
							\item They aren't usually stored, and get urinated out
							\item So, a steady supply is needed
							\item when protiens are broken down, there are left over nitrogens in the form of amonia, which is toxic, which needs to be evacuated in the urine
						\end{itemize}
				\end{itemize}
			\item \textbf{Carbohydrates} 
				\begin{itemize}
					\item Sugars, starches, glycogen
						\begin{itemize}
							\item Glycogen is the animal material form of starch
								\begin{itemize}
									\item most of which is found in the liver
									\item also a long chain molecule of glucoses
								\end{itemize}
						\end{itemize}
					\item Main source of energy
						\begin{itemize}
							\item starches and other polysaccharides get broken down to sugar
							\item sugar (esp. glucose) burned to make ATP
								\begin{itemize}
									\item carried in the blood to almost all of the cells
								\end{itemize}
						\end{itemize}
					\item plants build the starches during the day with excess glucose, so they have energy at night when they don't have the sun
				\end{itemize}
			\item \textbf{Lipids} 
				\begin{itemize}
					\item Fatty acids, triglycerides, cholesterol
					\item build cell membranes
					\item secondary source of energy
						\begin{itemize}
							\item much more enrgy-dense than carbohydrates
							\item used as a storage molecule
							\item have a lot more energy than carbs per pound
						\end{itemize}
				\end{itemize}
		\end{itemize}
	\item Energy
		\begin{itemize}
			\item All macro nutrients can be used for ATP productionn
			\item \textbf{METABOLISM} == convert food into energy or building blocks
			\item $C_6H_{12}O_6 + 6O_2 \to 6CO_2 + 6H_2O + 36ATP$ -- respiration (inverse of photosynthesis)
			\item ATP "density"
				\begin{itemize}
					\item Glucose = about 34 ATP per glucose molecule
						\begin{itemize}
							\item 6 carbons
						\end{itemize}
					\item Fatty acides = about 34 ATP per 4 carbons of the FA (fatty acid)
						\begin{itemize}
							\item Chain length can be between 6-22 carbons and up
							\item fatty acids only come in even numbers of carbon
						\end{itemize}
				\end{itemize}
			\item Calorie density:
				\begin{itemize}
					\item 4 Calories / gram
					\item 9 Calories / gram
					\item so, it's much more weight efficient to store exess energy as fat
					\item e.g. bird migration is powered by lipids to save weight
					\item \underline{C}al == 1000 \underline{c}al
					\item 1 calorie = $5-8\times 10^{19}$ ATP molecules
					\item So, 1g carbs = 4 Cal = 4000 cal = 200-320 sextillion ATPs
				\end{itemize}
		\end{itemize}
	\item Energy Balance
		\begin{itemize}
			\item If average energetic demand equals energetic input, all consumed energy is used for body funcntioning
			\item if \textit{input exceed demand}, suprlus energy is stored as \textbf{glycogen}  or \textbf{triglyceride}  if not used within hours or days
			\item Obesity -- one of the leading global causes of preventable eath
				\begin{itemize}
					\item 2.5 million/year
					\item Effects: heat disease, diabetes, arthritis, stroke, dimentia
				\end{itemize}
			\item if \textit{demand exceeds input}, stored glycogen and triglycerides are mobilized
			\item \textbf{Malnutrition} :
				\begin{itemize}
					\item Without sufficient calorie intake, body consumes fat resereves, then protein reserves
				\end{itemize}
		\end{itemize}
\end{itemize}

\end{document}
