\documentclass{article}
\usepackage[utf8]{inputenc}

\title{Week 14 Lecture 0}
\author{Jared Brannan }

\usepackage{natbib}
\usepackage{graphicx}
\graphicspath{ {./figures/} }
\usepackage{mathtools}
\usepackage{amsthm}
\usepackage{amsmath}
\usepackage{amssymb}
\usepackage{xcolor}
\usepackage{bbm}
\usepackage{bm}
\usepackage{physics}

% indent first line
\usepackage{indentfirst}
% one inch margins
\usepackage[margin=1.0in]{geometry}

\theoremstyle{definition}

\newcommand{\upRiemannint}[2]{
\overline{\int_{#1}^{#2}}
}
\newcommand{\loRiemannint}[2]{
\underline{\int_{#1}^{#2}}
}

\newtheorem{definition}{Definition}
\newtheorem{asside}{Asside}
\newtheorem{conjecture}{Conjecture}
\newtheorem{example}{Example}
\newtheorem{theorem}{Theorem}
\newtheorem{lemma}{Lemma}
\newtheorem{puzzle}{Puzzle}
\newtheorem{corollary}{Corollary}
\newtheorem{proposition}{Proposition}


\begin{document}
\maketitle

\section{Administrative drivel}
\begin{itemize}
	\item Answer key to exam 3 will be posted to canvas
	\item I'd promised flash cards -- I don't think I will be able to generate these for you before the exam. Send me an email (jared.brannan@wsu.edu) if you want a csv file of them, since that's how I use them. 
\end{itemize}

\section{Digestive system}
\begin{itemize}
	\item Today we'll wrap up, and move on to human ecological impact
	\item Small intestine -- duodenum
		\begin{itemize}
			\item Further (most) chemical breakdown in preperation for absorbtion
			\item duodenum detects acidity and uses bicarbonate (from the pancreus?) to neutralize the chyme (gets the ph to around 7)
			\item here chyme is broken down into it's basic molecular form (eg sacharides are broken into simple sugars)
			\item makes up the first foot/foot and a half of the small intestine
		\end{itemize}
	\item Small intnestinen
		\begin{itemize}
			\item about 5-5.5 meters long
			\item continued physical and enzymatic breakdown
			\item this is where \textbf{absorption} of nutrients from the chyme
			\item makes up the bulk of the small intestine to maximize the surface area for absorption to occur
			\item parts: jejunum and ileum
			\item walls are very thin for allowing greater absorption
			\item it's hypertrophied by finger like portrusions int the wall called \textbf{villi} which have folds in them called \textbf{microvilli} 
			\item the surface area is proportional to a tenis court
			\item simple sugars, amino acids, lipids, macronutrients all taken up by absorptive cells, then passed on to the capillaries and lymphatic system
				\begin{itemize}
					\item fats get preferentially taken up into the lymphatic system, since it's hydrophobic
				\end{itemize}
		\end{itemize}
	\item Large intestine (aka colon)
		\begin{itemize}
			\item No additional macronutrient absoption
			\item \textbf{Recover water} that was added earlier along with any remaining dissolved \textbf{minerals and vitamins} 
				\begin{itemize}
					\item lots of water added in chyme production
					\item energy is expended to pull minerals/vitamins
				\end{itemize}
			\item Withe water taken out the chyme is now called feces
			\item \textbf{Feces} build up against anal sphincter
				\begin{itemize}
					\item creates the urge to defecate
					\item regain coscious control
				\end{itemize}
		\end{itemize}
	\item \textbf{Vomiting} 
		\begin{itemize}
			\item typically tirggered by presence of toxic substances in food (e.g. bacterial toxins)
			\item close off the intestinal openig and squeeze down the stomach, reverse the peristaltic waves in the esophagus to send chyme back up
		\end{itemize}
	\item \textbf{Ulcers} 
		\begin{itemize}
			\item Holes in the GI tract (stomach and duodenum)
			\item Until recently, stress and food were blamed.
			\item Now, we know they're caused by bacteria
				\begin{itemize}
					\item figured out by Warren and Marshall
					\item 1982 heliobacter pylori discovered
					\item 1984 extreme measures\item 2005 Nobel prize inn Medicine
					\item Marshall drank the bacteria, got ulcers, then took antibiotics, and his ulcers cleared
				\end{itemize}
			\item Other risk factor -- ceertaain pain meds that include asprin and ibuprofen
		\end{itemize}
	\item youtube id 5ufESc1bK78 -- video tour of digestive system
\end{itemize}

\section{Global Climate Change and Human Impacts}

\begin{itemize}
	\item Been of growing concern since the 1980's, though we were aware that gas change in the atmosphere would change the climate
	\item If the earth didn't have an atmosphere, the earth would be -300 degreets
	\item Overview:
		\begin{itemize}
			\item The intergovernmental panel on climate change (IPCC) was esetablished in 1988 by the UN
				\begin{itemize}
					\item goal: to see if the change of composition of the atmosphere were going to cause problems
					\item in the 50's $CO_2$ increase measured
				\end{itemize}
			\item Panel includes experts in atmospheric and climatic science from around the world
				\begin{itemize}
					\item Done such that 1st world countries cannot dominate
				\end{itemize}
			\item In recognition ofo their efforts to spread "knowledge about man-made climate change," the IPCC was awarded the Nobel peace Prize in 2007
				\begin{itemize}
					\item They modeled the changes in climate to predict the future climate changes
				\end{itemize}
			\item  In its third report (2001), the IPCC concluded that the majority of the observed global warming was attributablel to human activites
				\begin{itemize}
					\item Not telling us what to do, just what's happening and some options of how to proceed
					\item what to do will be up to the polititians
				\end{itemize}
			\item if the temp increases by 2.0 degrees F, their will be consequences for living organisms
			\item In 2018 the IPCC determined warming is occurring faster than predicted a few years ago
			\item we have about 10 years to radically change the way we get energy, food, etc, otherwise it may be too late
				\begin{itemize}
					\item We know this by knowing exactly how much $CO_2$ we can afford to add to the atmosphere before things get bad
					\item so, this 10 year approximation is based on out current rates of gas emmision
				\end{itemize}
		\end{itemize}
	\item Atmospheric changes:
		\begin{itemize}
			\item Greenhouse effect (physical component)
			\item we've been producing green house gasses much more since we switched to fossil fules for producing goods during the industrial revolution
			\item doing so, changed the chemistry of the atmosphere
			\item even if we stopped producinng $CO_2$, we would still produce much more than humans did in the past just from respiration
		\end{itemize}
	\item Some other consequences that follow:
		\begin{itemize}
			\item sea-level changes
			\item human health consequences
				\begin{itemize}
					\item mosquitos will do better along with other pests
				\end{itemize}
			\item crop production will suffer
			\item altered habitats with  declines and redistribution of biodiversity
		\end{itemize}
\end{itemize}


							
\end{document}
