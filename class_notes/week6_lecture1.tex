\documentclass{article}
\usepackage[utf8]{inputenc}

\title{Week 6 Lecture 1}
\author{Jared Brannan }

\usepackage{natbib}
\usepackage{graphicx}
\usepackage{mathtools}
\usepackage{amsthm}
\usepackage{amsmath}
\usepackage{amssymb}
\usepackage{xcolor}
\usepackage{bbm}
\usepackage{bm}
\usepackage{physics}

% indent first line
\usepackage{indentfirst}
% one inch margins
\usepackage[margin=1.0in]{geometry}

\theoremstyle{definition}

\newcommand{\upRiemannint}[2]{
\overline{\int_{#1}^{#2}}
}
\newcommand{\loRiemannint}[2]{
\underline{\int_{#1}^{#2}}
}

\newtheorem{definition}{Definition}
\newtheorem{asside}{Asside}
\newtheorem{conjecture}{Conjecture}
\newtheorem{example}{Example}
\newtheorem{theorem}{Theorem}
\newtheorem{lemma}{Lemma}
\newtheorem{puzzle}{Puzzle}
\newtheorem{corollary}{Corollary}
\newtheorem{proposition}{Proposition}


\begin{document}
\maketitle

\section{Administrative drivel}
\begin{itemize}
	\item Nothing really.
\end{itemize}

\section{Anatomy and Physiology}
We're now going to cover organs and organ systems
	
\subsection{Skeletomuscular system:}
\subsubsection{Skeletal system}
\begin{itemize}
	\item 2 basic parts: Axial skeleton and Appendicular skelton
		\begin{itemize}
			\item Axial is centered around the spine and includes everything connected to the spine (spine, skull, ribcage)
				\begin{itemize}
					\item spine keeps the body erect
					\item mineral storage is in the bones, mostly calcium
					\item helps move the body around
					\item protects the heart and lungs
					\item protects the brain (the brain is easily damaged)
				\end{itemize}
			\item Appendicular is limbs + pelvis
		\end{itemize}
	\item The following bold bits are the parts of the Axial skeleton and their properties
	\item \textbf{skull}
		\begin{itemize}
			\item Protects the brain
			\item provides an entry for food
			\item mineral storage
			\item Only mamals can generate facial expressions for communication
			\item In humans, speach comes from the skull
		\end{itemize}
	\item \textbf{Spine} 
		\begin{itemize}
			\item Runs along the back of the body (dorsal)
			\item provides a fairly rigid system for hanging things on
				\begin{itemize}
					\item Lots of organs are connected to the spine to ensure they stay in position
				\end{itemize}
			\item Keeps the body errect
			\item Helps move the body around
			\item subparts: Cervical (neck), Thoracic (upper back, rib cage occurs here), Lumbar, Sacral Coxix vertebrae (The last 2 are in the pelvis, and the coxix is the tail that's been fused)
			\item the thoracic cavaty of the torso is surrounded by the ribcage
			\item mamalls all have 7 cervical vertebrae (including Giraffs!
			\item vertebrae are seperated by discs of cartilege that  allow it to flex (joints)
			\item the vertebrae and discs are hollow, surrounding the spinal chord  which connects the brain to the rest of the body
		\end{itemize}
	\item \textbf{Ribcage}
		\begin{itemize}
			\item Attached to the thoracic vertebrae by joints along the spine and sternum
			\item the flexibility of the ribcage helps with breathing
			\item Helps move the boy around and protects the heart and lungs
				\begin{itemize}
					\item Muscles attach the arms to the ribcage helping with movement
					\item there are 2 muscles between each pair of ribs that do the bulk of the work in breathing
				\end{itemize}
		\end{itemize}
	\item Clicker q: What's the function of the skull? Protect the brain
	\item The followingn bold items are the parts of the Appendicular skeleton
	\item \textbf{Arms and Legs} 
		\begin{itemize}
			\item Function: Helps move the body around
			\item Arm includes the clavical (colar bone) and the scapula (sholder blade)
			\item Legs include the pelvis
			\item keeps the body erect and stores minerals as well.
		\end{itemize}
	\item a few more boones to know:
		\begin{itemize}
			\item 209 bones in the ebody (most in the skull)
			\item The different vertebra
			\item skull: cranium, maxilla (upper jaw), dentary (lower jaw), orbit (eye socket, multiple bones), zygomatic arch (cheek bone, multiple bones), foramen magnum (the hole where the spinal chord exits the brain)
			\item bones of the arm: carpals, meta carpals, phalanges, radius, ulna, humerus
			\item parts of the leg: femur, patella, tibia, fibula, tarsals, metatarsals, phalanges, pelvic gridle
			\item \textcolor{red}{***SHOULD LOOK UP (DIAGRAM??)***}
		\end{itemize}
	\item Bones:
		\begin{itemize}
			\item Things to know:
				\begin{itemize}
					\item parts of a bone (in order on a long bone):
						\begin{itemize}
							\item Articular cartiledge (allows for smooth motion)
							\item spongy bone (toward the "ends" of long bone)
							\item Epiphyseal line
							\item Red bone marrow (in spongy bone)
							\item endosteum
							\item compact bone (makes up endosteum)
							\item medullary cavity
							\item yellow marrow (in cavity above)
							\item periosteum
								\begin{itemize}
									\item tough fibrous membrane that protects the bone, and produces new bone cells
								\end{itemize}
							\item nutrient artery
						\end{itemize}
					\item Location and function of:
						\begin{itemize}
							\item red marrow (blood cell formation (red), millions a day, since red blood cells have a 3 month lifespan)
							\item Yellow marrow (fat (lipid) storage)
							\item compact vs spongy bone (outside vs within ends respectively)
							\item cartilage at joints
						\end{itemize}
				\end{itemize}
			\item Bone -- structure:
				\begin{itemize}
					\item Bone has a structure that is grown and layed down peice by peice
					\item Haversian system (increases diameter):
						\begin{itemize}
							\item osteocyte -- bone cells! These grow bones (ostyoblasts when growing, osteocytes when the bone is done) 
							\item Synthesize layers of calcium layers in rings up and down the length of the bone, much like tree rings.
							\item there's a hollow cannal called the haversian canal where nerves and blood vessels run through the bone
							\item spongy bone surrounded by compact bone surrounded by articular cartilage with blood vessels running throughout
						\end{itemize}
				\end{itemize}
			\item Bone -- growth
				\begin{itemize}
					\item Length occurs at the ends of the bones at te epiphysis (the ends of the bone shaft)
					\item the epiphyses aren't fused to the caps till the bone is done growing
					\item Rather than adding layers to the entire surface as growth occurs in trees
					\item in early stages, bones are composed of collagin protein structure initially layed down, then calcium is impregnated into the collagin structure
					\item so, bone is a collagin matrix with osteocytes embedid and calcium minerals are stored
				\end{itemize}
			\item cell types:
				\begin{itemize}
					\item Osteoclasts: break bone apart and abosrb minerals (calcium) for reuse
					\item Osteoblasts deposit collagen / minerals crystallize around it (grow bones, less numerous in adults)
					\item Osteoocytes: osteoblasts that are found within older bones
				\end{itemize}
			\item rebuilding of bones:
				\begin{itemize}
					\item called ``bone remodelling
					\item osteoblasts originate inn the marrow
					\item growth factors are stored in bone
					\item osteoblasts build new bone tissue
					\item Hydroxy-appitite is the main constituant of bone
					\item bone density metric used to determine bone strength
				\end{itemize}
			\item bone repair:
				\begin{itemize}
					\item Osteoblasts and osteoclasts work to repair fractures
					\item bones are repaired in their current position, so if a fracture doesnt meet it will round off, if they meet at an offset, they'll fuse in that position
					\item If you allow flexibility as it heals, the bone will form a joint!
					\item repaired bones are stronger than the original bone!
				\end{itemize}
		\end{itemize}
\end{itemize}


\end{document}
