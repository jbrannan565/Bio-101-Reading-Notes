\documentclass{article}
\usepackage[utf8]{inputenc}

\title{Week 9 Lecture 1}
\author{Jared Brannan }

\usepackage{natbib}
\usepackage{graphicx}
\graphicspath{ {./figures/} }
\usepackage{mathtools}
\usepackage{amsthm}
\usepackage{amsmath}
\usepackage{amssymb}
\usepackage{xcolor}
\usepackage{bbm}
\usepackage{bm}
\usepackage{physics}

% indent first line
\usepackage{indentfirst}
% one inch margins
\usepackage[margin=1.0in]{geometry}

\theoremstyle{definition}

\newcommand{\upRiemannint}[2]{
\overline{\int_{#1}^{#2}}
}
\newcommand{\loRiemannint}[2]{
\underline{\int_{#1}^{#2}}
}

\newtheorem{definition}{Definition}
\newtheorem{asside}{Asside}
\newtheorem{conjecture}{Conjecture}
\newtheorem{example}{Example}
\newtheorem{theorem}{Theorem}
\newtheorem{lemma}{Lemma}
\newtheorem{puzzle}{Puzzle}
\newtheorem{corollary}{Corollary}
\newtheorem{proposition}{Proposition}


\begin{document}
\maketitle

\section{Administrative drivel}
\begin{itemize}
	\item exams will be graded by monday at the latest
	\item Revised paper due Nov, 8th at 11:59pm
		\begin{itemize}
			\item \underline{Turn on tracking} for the paper revision!
			\item the grade will be no lower than the first grade.
		\end{itemize}
\end{itemize}

\section{Anatomy and Physiology}
\subsection{Respiratory system}
\begin{itemize}
	\item Breathing -- control
	\item rate control
		\begin{itemize}
			\item There are monitors on the blood composition (main in the carated artery in the neck)
			\item Chemoreceptors in aorta and carotid arteries increase rate when $O_2$ drops too low, or $CO_2$ gets too height
			\item several other bain areas are also involved
		\end{itemize}
	\item normal breating -- 500 ml in and out (about 1 pint) == ``\textbf{tidal volume}"
		\begin{itemize}
			\item \textit{Forced}  inhalation can bring in additional air: around 3300mL in males, or 1900 mL in females
			\item \textit{Forced} exhalaltion can foce  out additional air: around 1000mL in males, or around 700mL in females
			\item even after forced exhalation, some residual air remains, stuck in incompressible spaces (e.g. trachea) or in alveoli, holding them open for a total of about 1100-1200mL of space
			\item this give the total lung air volume of 3200-6000mL or .75-1.5 gallons
			\item $O_2$ in the blood drops only by about a quarter when it passes through the tissues
		\end{itemize}
	\item \textbf{Respiration} 
		\begin{itemize}
			\item Beathing == pulling air in and out of the lungs
			\item \textbf{Respiration}  == exchanging gases between air and tissues
			\item \textbf{Respiration}  == exchange of gases between air in the alveoli and capillary blood surrounding the alveoli \textcolor{red}{Diffusion driven by difference in concentration} between gases in air and gases in blood
			\item gases  will independently diffuse down their gradients
			\item Recall from membranes -- $O_2$ and $CO_2$ are small and non-poolar so they move through membranes.
		\end{itemize}
	\item Gas exchange in an alveolus:
		\begin{itemize}
			\item blood comes from the heart into a capilary surrounding the alveoli, gas exchange ocurs, then blood is sent down a vein to the heart all oxygenated and with less $CO_2.$
			\item There are chemicals inside the alveolus that keeps it from getting stuck in a collapsed position
		\end{itemize}
	\item Gas transport -- oxygen
		\begin{itemize}
			\item $O_2$ has low solubility in liquids, and doesnt convert to other molecules (covalent bonds), but loves to bind to hemoglobin once it reaches
				\begin{itemize}
					\item ie, $O_2$ is not very reactive with anything in the blood till it gets to the hemoglobin's iron molecule 
				\end{itemize}
		\end{itemize}
	\item clicker q: what is the diaphragm? The muscle below the ribs that is involved in breathing
	\item Red blood cells AKA erythrocytes
		\begin{itemize}
			\item concentration of oxygen in the plasma is low, since much of it binds to hemoglobin, allowinng more oxygen to diffuse the plasma from the lungs
			\item Each cell carries about 200 million hemoblobin molecules
			\item oxygen bind to the \textbf{heme} group -- it contains $Fe^{2+}$ (iron ions)
				\begin{itemize}
					\item hence the red color
				\end{itemize}
			\item Each hemoglobin can bind $4O_2.$
		\end{itemize}
	\item as transport -- carbon dioxide
		\begin{itemize}
			\item Higher (but still low) solubility in liquids than oxygen
			\item Redily forms \textit{carbonic acid}  in water, breaks into $H^+$ and \textit{bicarbonate} (This is the form that $CO_2$ is caried in the blood (in the plasma))
			\item $CO_2 + H_2O \to H_2CO_3 \to H^+ + HCO_3^-$
			\item $CO_2$ is combined with $H_2O$ by an enzyme
			\item at the lungs, this proocess reverses
			\item  The $H^+$ ion tends to interact with hemoglobin to knock off the $O_2$ from the hemoglobin (from increased Ph)
			\item $CO_2$ can also bind to hemoglobin
		\end{itemize}
	\item recall, in mitochondra: $C_6H_{12}O_6 + 6O_2 \to 6CO_2 + 6H_2O + 36ATP$
		\begin{itemize}
			\item $O_2$ is required to get the most ATP possible, $CO_2$ is wast product from this process.
		\end{itemize}
	\item so, aside from breathing, resperation is passive!
	\item \underline{Carbon monoxide} poisoning:
		\begin{itemize}
			\item $CO$ is produced by gasoline engines, power tools,, poorly functioning furnaces / stoves
				\begin{itemize}
					\item Colorless, odorless -- hard to detect
					\item preferentially binds to hemoglobin, so not as much $O_2$ can bind, thus, tissues are deprived of oxygen.
				\end{itemize}
		\end{itemize}
	\item \underline{Diseases}
		\begin{itemize}
			\item Infection and inflamation
				\begin{itemize}
					\item The lungs are 100\% humidity, and warm, which is a perfect place for infectious organisms to thrive
					\item Fumes from cars can make $O_3$, an iritant for tissues, iritating the tissues, stimulating an imune response causing inflamation
				\end{itemize}
			\item Problems with air transport (in or out)
				\begin{itemize}
					\item Constrictions or restrictions
					\item choking
				\end{itemize}
			\item Problems with gas diffusion
		\end{itemize}
	\item \textbf{Asthma}  -- inflamation of the bronchioles
		\begin{itemize}
			\item iritation of the lining of the bronchi, causing the muscles in the walls to constrict, reducing the diameter of the passageway
			\item mucus is released
		\end{itemize}
\end{itemize}






\end{document}
