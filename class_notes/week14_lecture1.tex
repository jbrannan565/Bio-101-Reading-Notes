\documentclass{article}
\usepackage[utf8]{inputenc}

\title{Week 14 Lecture 1}
\author{Jared Brannan }

\usepackage{natbib}
\usepackage{graphicx}
\graphicspath{ {./figures/} }
\usepackage{mathtools}
\usepackage{amsthm}
\usepackage{amsmath}
\usepackage{amssymb}
\usepackage{xcolor}
\usepackage{bbm}
\usepackage{bm}
\usepackage{physics}

% indent first line
\usepackage{indentfirst}
% one inch margins
\usepackage[margin=1.0in]{geometry}

\theoremstyle{definition}

\newcommand{\upRiemannint}[2]{
\overline{\int_{#1}^{#2}}
}
\newcommand{\loRiemannint}[2]{
\underline{\int_{#1}^{#2}}
}

\newtheorem{definition}{Definition}
\newtheorem{asside}{Asside}
\newtheorem{conjecture}{Conjecture}
\newtheorem{example}{Example}
\newtheorem{theorem}{Theorem}
\newtheorem{lemma}{Lemma}
\newtheorem{puzzle}{Puzzle}
\newtheorem{corollary}{Corollary}
\newtheorem{proposition}{Proposition}


\begin{document}
\maketitle

\section{Administrative drivel}
\begin{itemize}
	\item Clicker serial number assignment on Canvas
	\item No clicker qs today -- no base
	\item final on the 16th at 4:30 in the usual classroom
	\item Review sheet should be posted tonight
		\begin{itemize}
			\item The exam will be 45-50 questions
			\item 2 hours to complete
		\end{itemize}
\end{itemize}

\section{Climate change -- continued}
\begin{itemize}
	\item We'll finish climate change today -- hopefully
	\item Temperature change
		\begin{itemize}
			\item Warmer in the last few decades, almost certainly human caused
			\item fosil fuels release carbon that's been tied up in rocks for thousands of years into the atmosphere
			\item the amount of carbon the atmosphere is relatively constant, at least for the last 10s of millions of years
			\item burning fosil fuels has changed this
			\item since the 80's we've seen a steady rise in temp which is associated with increase in $CO_2$
		\end{itemize}
	\item Some evidence and patterns
		\begin{itemize}
			\item There is a tight relationshep between temperature and $CO_2$ concentration
				\begin{itemize}
					\item How do they know the $CO_2$ concentration 160k years ago?
						\begin{itemize}
							\item Looking at ice cores!
						\end{itemize}
				\end{itemize}
			\item Historic
				\begin{itemize}
					\item Steady $CO_2$ till the mid 1980's when exponential growth began
				\end{itemize}
			\item More recent
				\begin{itemize}
					\item $CO_2$ has been exponentially increasing by industry output, and has stopped during 3 periods in the 20th century
					\item The use of electricity is accelerating, increasing $CO_2$ production
				\end{itemize}
			\item very recent
				\begin{itemize}
					\item $CO_2$ production Accelerating over the last 50 years
				\end{itemize}
			\item $C$ 14 short half life, not present in fossile fuels, so we can measure its concentration  to see how much is from fosil fuels
			\item The $C$ 14 concentration in the atmosphere has decreased with the burning of fossil fuels depleted in $C$ 14
		\end{itemize}
	\item Coal -- major contributor of $CO_2$
		\begin{itemize}
			\item In volume of $CO_2$ concentration, coal is one of the largest contributor
			\item Used for most electricity
			\item natural gas makes about 1/3 the $CO_2$ that coal makes
			\item it costs a lot to make power plants that are clean, so we have a lot coal generators still
			\item Now, coal is substantially more expensive than other sources, as of fairly recently
		\end{itemize}
	\item It's not only $CO_2$ 
		\begin{itemize}
			\item Methane ($CH_4)$, Chlorofluorocarbons (CFCs), Nitrous oxide ($N_2O$)
				\begin{itemize}
					\item Methane is about 20 times more potent as a greenhouse gas than $CO_2$ by weight
					\item CFCs are rare -- used in refrigerants -- made the hole in the ozone, so it's outlawed as of 30 ish years ago
				\end{itemize}
		\end{itemize}
	\item Greenhouse effect: how it works
		\begin{itemize}
			\item most of what happens is 
				\begin{itemize}
					\item Sun emits energy through the atmosphere (some visible, most invisible)
					\item Some bounces off and goes into space
					\item some comes through and is
						\begin{itemize}
							\item absorbed by atoms in the atmosphere or
							\item absorbed by atoms on the land 
							\item absorbtion transforms the energy into heat -- thermal energy
							\item absorption heats the earth
							\item some bounces back out into space
						\end{itemize}
					\item Some of the heat is let out as radiation cooling the earth
					\item If the amount of radiation coming in is the same as going out, the earth's temperature will be at equilibrium (constant)
					\item greenhouse gasses absorb some of the radiation on it's way out of the atmosphere, turning it back into heat, so energy gets trapped, breaking the equalibrium, heating the earth's atmosphere
				\end{itemize}
		\end{itemize}
	\item We're in an interglacial period, so the the temp/$CO_2$ levels are high just from normal changes in temp/$CO_2$, but now we're well above the norm and at a much faster rate of increase than has ever been seen
	\item things to do: drive less (walk, bike, take the bus), avoid air travel when possible (airplanes dump a lot of $CO_2$)
	\item Things goverments can do: finance power grids, tax carbon, regulate greenhouse gas emition
	\item half life of $CO_2$ is long -- 120 years -- so it takes a long time to get out
	\item Most of the warming is at the far north, and over land
\end{itemize}


							
\end{document}
