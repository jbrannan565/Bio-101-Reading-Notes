\documentclass{article}
\usepackage[utf8]{inputenc}

\title{Week 14 Lecture 2}
\author{Jared Brannan }

\usepackage{natbib}
\usepackage{graphicx}
\graphicspath{ {./figures/} }
\usepackage{mathtools}
\usepackage{amsthm}
\usepackage{amsmath}
\usepackage{amssymb}
\usepackage{xcolor}
\usepackage{bbm}
\usepackage{bm}
\usepackage{physics}

% indent first line
\usepackage{indentfirst}
% one inch margins
\usepackage[margin=1.0in]{geometry}

\theoremstyle{definition}

\newcommand{\upRiemannint}[2]{
\overline{\int_{#1}^{#2}}
}
\newcommand{\loRiemannint}[2]{
\underline{\int_{#1}^{#2}}
}

\newtheorem{definition}{Definition}
\newtheorem{asside}{Asside}
\newtheorem{conjecture}{Conjecture}
\newtheorem{example}{Example}
\newtheorem{theorem}{Theorem}
\newtheorem{lemma}{Lemma}
\newtheorem{puzzle}{Puzzle}
\newtheorem{corollary}{Corollary}
\newtheorem{proposition}{Proposition}


\begin{document}
\maketitle

\section{Administrative drivel}
\begin{itemize}
	\item Final paper submission grades are on canvas
	\item Clicker points should be up tomorrow
\end{itemize}

\section{Climate Change}
\begin{itemize}
	\item Most of the warming happens over land and at high latitude
		\begin{itemize}
			\item This results in the north pole melting
			\item Frozen water reflects more radiation back into space than liquid water
			\item Melting increases the ocean volume  -- sea level rise
				\begin{itemize}
					\item only from land ice melting, since ice in the water already displaces how much it would if it were liquid
				\end{itemize}
			\item Greenland has a lot of ice that's like 2 miles thick, and if it melts it will rise sea level by 6 feet -- that's a lot
		\end{itemize}
	\item The goal is too keep the temp bellow a 2.5 decrees C change from norm
	\item Glaciers not on the poles are also melting
		\begin{itemize}
			\item most of the water that was in the glaciers ends up in the ocean (evenually)
			\item e.g. kilimanjaro has lost most of its ice over the last 100 years
			\item Ice and snow contributes a lot to economy (skiing seasons are getting shorter)
			\item Rivers that are fed by snow and ice are losing volume, leading to water shortages 
				\begin{itemize}
					\item e.g. colorado river feeds arizona, california, etc.
					\item If it rains instead of snows on the mountain in the winter, the summer melts won't be as big
				\end{itemize}
		\end{itemize}
	\item The sea level is rising 1.5-2.2 cm per decade (about an inch
		\begin{itemize}
			\item this rate is accelerating
			\item this will lead to people getting displaced
				\begin{itemize}
					\item e.g. venice, london
					\item There are entire countries where all people live at 6ft above sea level
				\end{itemize}
		\end{itemize}
	\item not only does increased temperatures melt land ice
		\begin{itemize}
			\item warmer water occupies more volume than cool water -- this increases sea levels as well
		\end{itemize}
	\item The entire coast of florida will flood with a 1 meter sea level rie
		\begin{itemize}
			\item further, huricane damage will be greater
		\end{itemize}
	\item first world countries can deal with these problems (mostly) -- 2nd and 3rd really cant
	\item most people live within 100 miles of a coast -- humans like living near water
	\item Changes in precipitatoin due to increasing temps
		\begin{itemize}
			\item Rainfall is increasing in some areas!
			\item This can lead to flooding, mudslidess, etc
				\begin{itemize}
					\item Elivated soil temps will make them drier, even though there is more rain, so crop raising will be harder
				\end{itemize}
			\item Storms will be bigger -- Hurricanes and Typhons
				\begin{itemize}
					\item More energy in the atmosphere leads too higher winds
				\end{itemize}
		\end{itemize}
	\item Species' Rnages 
		\begin{itemize}
			\item Climate change changes where species can survive
			\item Butterflies and moths on the west coast are making mass migration out of their usual areas
			\item This also applies to crop plants, farm animals, and other species that we are dependent on
			\item e.g. sugar maple trees are having to be grown futher north with temp increase
				\begin{itemize}
					\item This migration may require an assist
				\end{itemize}
			\item many species wont adjust fast enough, and will go extinct
				\begin{itemize}
					\item e.g. polar bears and walruses are dependent on the water ice, so they will likely go extinct
				\end{itemize}
		\end{itemize}
	\item Desertification -- iappropriate land use\
		\begin{itemize}
			\item compounded by
				\begin{itemize}
					\item Overgrazing
					\item poor aggriculture practices
				\end{itemize}
			\item increased temperature will reduce soil moisture retention
			\item higher temps alone will reduce crop yeilds even if desertification isn't a problem, locally
		\end{itemize}
	\item Human land use + climate change has massive ecological impact
		\begin{itemize}
			\item Greater than their solo controbutions
			\item There are interactoins (statistical sense), impact is more than an addition of two independent effects
			\item Hbitat loss has been extensive -- especially conversoin to agriculture and urban/suburban landscapes, especially near water
			\item we're losing a lot of land area that could be used for agg to suburban sprawl
		\end{itemize}
	\item These problems can be reversed:
		\begin{itemize}
			\item IPCC -- guidance on Climate change (replace most fossil fuels, change agg practices, change how we build and maintain structures, etc.)
			\item BUT \textcolor{red}{\textbf{TIME IS CRITICAL}}  -- we need to act within the next decade or two to prevent irreversible damage
		\end{itemize}
	\item Political statement from prof -- the last 4 years where a real shit show as far as adressing climate change 
	\item Nuclear is only a temp solution -- if we switched to 100\% nuclear we'd only have about 45 years of fuel
	\item There are many deniers, and people who are worried about the cost, but we won't damage the enviornment/society by fixing these things. The worst that will occur is some people won't be as wealthy
\end{itemize}

\section{Biodiversity decline -- the 6th mass extinction}
\begin{itemize}
	\item This is as big of a problem as climate change
		\begin{itemize}
			\item There's a complex network of species dependencies, and if this network is changed too much, the castcade could be catistrophic
			\item this includes humans
		\end{itemize}
	\item The evidence is becoming grim 
		\begin{itemize}
			\item Fisheries in the Asia-Pacific will have no fish in the oceans by 2048...
			\item More than half of the world's fisheries are harvestedd at or beyond capacity
			\item We tend to harvest a desirable species until it declines or crashes
		\end{itemize}
	\item This is a case of the ``Tragedy of the Commons" -- free resources tend to get over exploited by humans to outcompete each other
		\begin{itemize}
			\item E.g.s
				\begin{itemize}
					\item Pllution of air and water
					\item depletion of auqifers
				\end{itemize}
		\end{itemize}
	\item HIPPO -- categories of species decline ( in order of effect)
		\begin{itemize}
			\item Habitat loss
				\begin{itemize}
					\item Almost everywhere
					\item bigest driver of drop in biodiversity
					\item deforestization, ag, suburban
				\end{itemize}
			\item Invasive species
				\begin{itemize}
					\item Compete with native species
					\item invasive species tend to escape predators, so they outcompete native
				\end{itemize}
			\item Pollution
				\begin{itemize}
					\item  many forms
					\item garbage in the ociean
				\end{itemize}
			\item Population -- human (over population)
				\begin{itemize}
					\item Growing rapidly
					\item acceleration with little sign of slowing down...
					\item 60 years ago we were at 3 billion people...
					\item we have surpassed the sustainable number...
					\item birth rates are coming down, but not fast enough...
				\end{itemize}
			\item Over harvesting
				\begin{itemize}
					\item Cut too many trees, kil to many terestrial animals, ooverfish
				\end{itemize}
		\end{itemize}
	\item we are at 1k to 10k species lost per year
	\item Solution -- spcecies area curve
		\begin{itemize}
			\item function of the number of species and land available
			\item Setting asside Half of all land and water area will save 85\% of species
			\item needs to be addressed at a governmental and individual level
				\begin{itemize}
					\item individuals can set aside some of their land to run wild
				\end{itemize}
		\end{itemize}
\end{itemize}

							
\end{document}
