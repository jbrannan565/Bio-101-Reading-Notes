\documentclass{article}
\usepackage[utf8]{inputenc}

\title{Week 2 Lecture 0}
\author{Jared Brannan }

\usepackage{natbib}
\usepackage{graphicx}
\usepackage{mathtools}
\usepackage{amsthm}
\usepackage{amsmath}
\usepackage{amssymb}
\usepackage{xcolor}
\usepackage{bbm}
\usepackage{bm}
\usepackage{physics}

% indent first line
\usepackage{indentfirst}

\theoremstyle{definition}

\newcommand{\upRiemannint}[2]{
\overline{\int_{#1}^{#2}}
}
\newcommand{\loRiemannint}[2]{
\underline{\int_{#1}^{#2}}
}

\newtheorem{definition}{Definition}
\newtheorem{asside}{Asside}
\newtheorem{conjecture}{Conjecture}
\newtheorem{example}{Example}
\newtheorem{theorem}{Theorem}
\newtheorem{lemma}{Lemma}
\newtheorem{puzzle}{Puzzle}
\newtheorem{corollary}{Corollary}
\newtheorem{proposition}{Proposition}


\begin{document}
\maketitle
\section{Administrative drivel}
\begin{itemize}
	\item clickers aren't working today
	\item You can use your favorite scientist
\end{itemize}

\section{More on classification}
\subsection{Characteristics of \textit{Family} \textbf{Hominidae}}
\begin{itemize}
	\item 3 genera, 5 species 
	\item most of the non-human primates aren't doing so hot
	\item Homonids:
		\begin{itemize}
			\item Pongo (orangutan)
			\item Homo (humans)
			\item Pan troglodytes (chimp)
			\item pan paniscus (bonobo)
			\item gorilla
			\item lowland gorillas
		\end{itemize}
	\item This list is much larger if you include the extict genera and species
\end{itemize}

\subsection{Charactristics of \textit{Genus} homo}
\begin{itemize}
	\item even larger brains than others in the family
	\item Reduced face and jaw face
	\item increased reliance on social/cultural interactions
	\item use of tools and fire
		\begin{itemize}
			\item tool usage is not unique to humans, but fire is (though not all humans have done so historically)
		\end{itemize}
	\item hands adept at complex manipulation
		\begin{itemize}
			\item lots of mamals have fewer toes/fingers (horses walk on 1 for example)
			\item 5 fingers/toes is the ``ancestral condition"
			\item The thing that makes human's unique is the dexterity, fine motor coordination, aposable thumbs, etc.
		\end{itemize}
	\item Habitually bipedal
		\begin{itemize}
			\item never though of being bipedal as a ``habit"
			\item chimps and other hominids can walk a short distance, and can knuckle walk, but aren't habitually bipedal
			\item this makes it easy to cover large distances on foot, since it frees up the hands to carry things, including young and tools, between food sources.
			\item this made it advantageous to only have 1 child at a time till each child can walk on it's own, so populations grew slowly till agriculture hit.
				\begin{itemize}
					\item no longer needed to be nomatic
					\item don't have to move around, so you can have a child every year, leading to the exponential growth in human populations. (this started 8-12 thousand years ago, the next spike in reproduction rate happened in mid 1800s from industry).
				\end{itemize}
		\end{itemize}
\end{itemize}

\subsection{classifications: know this}
(Left is the human category, you should know the characteristics of them as well (see summary slide))
\begin{itemize}
	\item domain -- eukarya
	\item kingdom -- animalia
	\item phylum -- chordata
		\begin{itemize}
			\item sub-phylum -- vertebrata
		\end{itemize}
	\item class -- mamalia
	\item order -- primate
	\item family -- hominidae
	\item genus -- homo
	\item species -- sapiens
		\begin{itemize}
			\item subspecies -- sapiens
		\end{itemize}
\end{itemize}
Species name: Homo sapiens sapiens
\begin{itemize}
	\item generally, (for our purposes) we say 2 animals are in the same species if they can reproduce, but this is not the full story for biologists.
\end{itemize}

\textbf{Q:} Are humans descended from chimpanzees? \textit{NO!!}, however, they share a recent common ancestor.

\section{Human evolutionary history}
We'll mostly cover the major branches -- this is not a complete coverage.
\begin{itemize}
	\item humans have been on their own for about 6 million years
\end{itemize}
\subsection{~20 million years ago}

\begin{itemize}
	\item Hominidae, the Great Apes, apeared
	\item Evolved likely in the african savana
		\begin{itemize}
			\item Arid and dry, but can support trees
			\item trees are sparse, since large animals knock them over
			\item texas was once a savana when there were mastadons there
		\end{itemize}
	\item standing up freed hands, and made it easier to see further
\end{itemize}

\subsection{~7-8 million years ago}
\begin{itemize}
	\item Human-like group seperated from great apes
		\begin{itemize}
			\item First Orangutan, then gorilla, then chimp lineages split off
		\end{itemize}
	\item Human ancestors moved into drier areas outside of the equatorial jungles
	\item last common ancestor with Pan: Ouranopithecus 
	\item they were near the equator and adapted to dry grasslands
\end{itemize}




\end{document}
