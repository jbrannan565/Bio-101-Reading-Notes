\documentclass{article}
\usepackage[utf8]{inputenc}

\title{Week 3 Lecture 0}
\author{Jared Brannan }

\usepackage{natbib}
\usepackage{graphicx}
\usepackage{mathtools}
\usepackage{amsthm}
\usepackage{amsmath}
\usepackage{amssymb}
\usepackage{xcolor}
\usepackage{bbm}
\usepackage{bm}
\usepackage{physics}

% indent first line
\usepackage{indentfirst}

\theoremstyle{definition}

\newcommand{\upRiemannint}[2]{
\overline{\int_{#1}^{#2}}
}
\newcommand{\loRiemannint}[2]{
\underline{\int_{#1}^{#2}}
}

\newtheorem{definition}{Definition}
\newtheorem{asside}{Asside}
\newtheorem{conjecture}{Conjecture}
\newtheorem{example}{Example}
\newtheorem{theorem}{Theorem}
\newtheorem{lemma}{Lemma}
\newtheorem{puzzle}{Puzzle}
\newtheorem{corollary}{Corollary}
\newtheorem{proposition}{Proposition}


\begin{document}
\maketitle

\section{Administrative drivel}
The exam will cover through today's material, sans lipids

\section{More on Biochistry: The Chemistry of Life}
\subsection{Atoms interact in different ways}
\begin{itemize}
	\item Atoms for \textit{Chemical bonds}  between each other based around their electrons
	\item Atoms do chemistry in order to fill up the outmost energy level in the atom (for most elements in LIFE that is 8 electrons, by H (hydrogen) needs only 2)
	\item Usually, for each proton there is an electron in the atom, so atoms are electrically neutral.
		\begin{itemize}
			\item this can be changed by adding or removing energy from the system
		\end{itemize}
	\item e.g.
		\begin{itemize}
			\item Hydrogen (H) requires 1-more electron to fill it's outermost energy level
			\item Oxygen requires 2-more electrons to fill it's outmost energy level (it has 6 in its outer shell usually, and it wants to have 8 in it's outmost shell)
			\item If we have 2 H, and 1 O in the states above, the can share electrons to get up to their desired number (This is an example of a covalent bond)
		\end{itemize}
	\item in the periodic table, in a given column tend to form the same types and number of bonds, since they have the same number of outermost elecctrons in their ``default state". i.e. the outermost shell of electrons has the same number of electrons among the atoms in a given column.
	\item The noble gasses (right most in the periodic table) have full outer shells in their default state, so they don't like to bond
\end{itemize}

\subsection{Ionic bonds - between ions}
\begin{itemize}
	\item Atoms will give up one of their electrons to another to fill their outermost shell. 
	\item \textbf{Salts} do this!
		\begin{itemize}
			\item Sodium (Na) has 1 electron in outer shell
			\item Chlorine (CL) has a "hole" (7 electrons in valence shell)
			\item Na gives it's electron to Cl
			\item the result is 2 \textbf{ions}: Cl is negatively charged ($Cl^-$) and Na is positively charged ($Na^+$ )
		\end{itemize}
\end{itemize}
\subsection{Covalent bonds- electron is shared}
\begin{itemize}
	\item characterized by \textbf{sharing}  electrong
	\item Molecules are chemicals that are atoms combined by covalent bonds
	\item very strong bond
	\item e.g.
		\begin{itemize}
			\item Water
			\item methane (natural gass)
			\item oxygen ($O_2$ )
		\end{itemize}
\end{itemize}

\subsection{Polarity, seperation of electric charge}
\begin{itemize}
	\item due to unequla sharing of electrons between two atoms
	\item now, opposite charges between molecules attract
	\item e.g.
		\begin{itemize}
			\item in water ($H_2 )$, oxygen likes to hoard the electron, making the oxygen end negative, and the side that the hydrogen live on is negative in the molecule.
		\end{itemize}
\end{itemize}

\subsection{Hydrogen bonds, relatively weak}
\begin{itemize}
	\item e.g. 2 water molecules can be bonded along their oposite chage (oxygen side of one water molecule bonds to the hydrogen end of the other water molecule, since the O side has a negative charge, and the H side has a negative charge)
	\item Happens between neighboring molecules
	\item Happens between polar molecules (see the previous section)
	\item (usually forms a latice with Xs made up of O in the middle surrounded by 4 Hs)
	\item easily broken, but happen everywhere, and are very ``usefull" in life molecules
	\item more generally, we call charge bonds like these ``polar bonds"
\end{itemize}

\subsection{More about water}
\begin{itemize}
	\item It's got good properties
		\begin{itemize}
			\item Works as a solvent for life! (due to mild polarity)
		\end{itemize}
	\item living things are mostly water (by mass)
	\item The polarity of water means polar and charged atoms and molecules can dissolve in it
		\begin{itemize}
			\item e.g. salt
			\item water surrounds the $Cl^-$ and $Na^+$ (with oposite charge pole facing the respective molecule) untill there is no more water molecules, or no more salt.
		\end{itemize}
	\item solvents allow cells to bring things in and out of themselves
\end{itemize}

\definition \textit{Molecule.} a formation of more than 1 atom that has covalent bonds between the atoms (usually)


\subsection{Important organic molecules}
\begin{itemize}
	\item Carbs
	\item Lipids
	\item ...
\end{itemize}

\subsubsection{Carbohydrates}
\begin{itemize}
	\item substances found in starchy plant materials (usually)
	\item Glucose
		\begin{itemize}
			\item 6 carbons in a ring ish thing, with hydrogens and oxygens
			\item the product of plant's photosynthesis
			\item this is the sugar in your blood
		\end{itemize}
	\item all 1 to 2 to 1 ratio of carbon, hydrogen, oxygen respectively
	\item Monosaccharides -- simple sugars
		\begin{itemize}
			\item either 5 or 6 carbons
			\item differ in shape, structure, arrangement
			\item glucose, fructose, etc
		\end{itemize}
	\item Disaccharides
		\begin{itemize}
			\item composed of 2 simple sugars
			\item e.g. lactose (milk sugar), sucrose (table sugar)
			\item lactose is galactose + glucose
			\item glucose + fructos = sucrose
			\item sugars are hooked together on their respective oxygens
		\end{itemize}
	\item sugar names end in ``ose"
	\item Polysaccarides (not sugars!)
		\begin{itemize}
			\item starch -- most common poly... in plants
				\begin{itemize}
					\item made up of the composition of more than 2 simple sugars
					\item sugars are hooked together on their respective oxygens
					\item this is long term energy storage for plants
					\item starch in some seeds is to provide energy for the seed to germinate
					\item most fruits have a lot of starch, some roots do too
					\item when fruits become ripe, the fruit breaks down the starches into sugars which attract other animal to distribute the seeds of the plant.
				\end{itemize}
			\item Glycogen
				\begin{itemize}
					\item characterized by many branching chains, instead of single chains in the starches.
					\item the way humans store energy (most vertebrates as well) store it in their liver and muscles for long term energy
					\item too much glycogen will get converted to fat (lipids)
				\end{itemize}
		\end{itemize}
\end{itemize}

\subsubsection{Lipids}
\begin{itemize}
	\item super energy dense molecules
	\item stores long term excess energy
	\item this can be a problem in humans, since it can be hard to get rid of.
	\item consist of fats and oils
	\item key components:
		\begin{itemize}
			\item glycerol + fatty acids
			\item when you have 3 fatty acid chains that are attached to the glycerol ``backbone" you get a \textbf{triacylglyceral}  or \textbf{triglyceride} 
			\item fatty acid: long chain of carbons with a bunch of hydrogens, the end OH tends to break off and bonds with the clycerol making water and a lipid.
			\item fatty acids are the basic lipid, and there are many different kinds, and can be combined into a wide veriety of triglycerides.
			\item almost all fatty acids have an even number of carbons
		\end{itemize}
	\item Triglicerides:
		\begin{itemize}
			\item Main long-term energy storage in humans
			\item lipids are stored in fat cells in a vessicle
			\item migratory birds in late summer will double their weight in fat and burn it off in migration
		\end{itemize}
	\item fatty acids vary
		\begin{itemize}
			\item straight if all of the carbons are single bond
				\begin{itemize}
					\item tend to be solid at room temp, since they can line up with each other
					\item called saturated
					\item tend to make fat deposits in the arteries long term, which is noo good.
				\end{itemize}
			\item you get a kink in the acid if there's a double bond in the carbon chain
				\begin{itemize}
					\item plants tend to make these
					\item tend to be liquid at body temp, since the acide don't line up as easily
					\item called unsaturated
				\end{itemize}
		\end{itemize}
\end{itemize}



\end{document}
