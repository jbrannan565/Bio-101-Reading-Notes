\documentclass{article}
\usepackage[utf8]{inputenc}

\title{Week 9 Lecture 0}
\author{Jared Brannan }

\usepackage{natbib}
\usepackage{graphicx}
\graphicspath{ {./figures/} }
\usepackage{mathtools}
\usepackage{amsthm}
\usepackage{amsmath}
\usepackage{amssymb}
\usepackage{xcolor}
\usepackage{bbm}
\usepackage{bm}
\usepackage{physics}

% indent first line
\usepackage{indentfirst}
% one inch margins
\usepackage[margin=1.0in]{geometry}

\theoremstyle{definition}

\newcommand{\upRiemannint}[2]{
\overline{\int_{#1}^{#2}}
}
\newcommand{\loRiemannint}[2]{
\underline{\int_{#1}^{#2}}
}

\newtheorem{definition}{Definition}
\newtheorem{asside}{Asside}
\newtheorem{conjecture}{Conjecture}
\newtheorem{example}{Example}
\newtheorem{theorem}{Theorem}
\newtheorem{lemma}{Lemma}
\newtheorem{puzzle}{Puzzle}
\newtheorem{corollary}{Corollary}
\newtheorem{proposition}{Proposition}


\begin{document}
\maketitle

\section{Administrative drivel}
\begin{itemize}
	\item New due date for term paper revisions: Due Mon, Nov. 8 11:59pm
	\item Date for \underline{\textbf{final exam}} Thurs Dec 16th at 4:30pm
\end{itemize}

\section{Anatomy and Physiology}
\subsection{Respiratory system}
\begin{itemize}
	\item The surface area in the lungs is larger than the average house
	\item Parts of the alveoli:
		\begin{itemize}
			\item Alveolar duct, blood vessels, lumen of bronchiole, alveolar sac
		\end{itemize}
	\item in the lungs in the alveoli, the boundary between the alveoli and the capilaries is 2 cells thick (1 cell thick in the alveoli and 1 cell thick in the capilary)
		\begin{itemize}
			\item once again, diffusion over short distances is quick, long distances are slow
		\end{itemize}
\end{itemize}
\subsubsection{Breathing}
\begin{itemize}
	\item Clicker q: what is the beneefit of lots of tiny little alveoli for gags exchange instad of just a single big sack of a lung? Lots of alveoli means larger surface area for gas exchange.
	\item When we breath, we're moving air (a fluid)
	\item Fluids flow from areas of high pressure to areas of low presure, i.e. fluids flow down a pressure gradient
	\item Breathing = adjusting lung pressure so that air flows down the pressure gradient into or out of the lungs.
	\item We adjust the pressure by adjusting the volume of the thorasic cavity
	\item Pressure:
		\begin{itemize}
			\item \textbf{Atmospheric pressure:} 760 mmHg (sea level)
			\item \textbf{Inhilation}: making pressure in alveoli lower than 760 mmHG, an air will flow in
			\item \textbf{Exhalation}: Make pressure in alveoli higher than 760 mmHg, and air will flow out
			\item How to adjust pressure in the alveoli:
				\begin{itemize}
					\item \underline{Adjust volume} and pressure follows (increase volume, pressure decreases, decrease volume pressure  increases)
				\end{itemize}
		\end{itemize}
	\item Inhilation:
		\begin{itemize}
			\item To cause air to flow into the lungs, lower the air pressure in the lungs below atmospheric pressure
			\item it lower pressures in the lungs, increase the volume of the lungs
		\item Muscles (between ribs, external \textbf{intercostal} muscles) surrounding lungs pull outwards (and up) and downwards (diaphragm muscle)
			\item only mamals have a diaphragm: it coverse the botom of the ribcage
			\item a ``stitch" in the side is a cramp in the diaphragm!
		\end{itemize}
	\item Exhalation:
		\begin{itemize}
			\item done by raising the pressure in the lungs by decreasing the volume
			\item Muscles surrounding the lungs relax, elastic nature of lungs shrinks their volume
			\item and, forced exhalation also involves \textbf{internal intercostal} muscles
			\item this control allows us to talk, sing, blow out candles, etc.
			\item diaphragm relaxes into the domed position, helping decrease the volume of the lungs
		\end{itemize}
	\item note: the diaphragm is not esential to breath, but helps with controlled breathing
		\begin{itemize}
			\item Birds and reptiles breath just fine without them
		\end{itemize}
	\item CLicker q: in order to exhale the pressure in the lungs has to be greater than 760 mmHg
	\item \textbf{Breathing control} 
		\begin{itemize}
			\item Unconscious control over the diaphragm
			\item controlled by the brain stem -- majority of nervous control
			\item Generally gives aboout a 2 second contractiono, followed by 3 seconds of relaxtion, a bit on the fast side but generally 12-18 breathes/min
			\item Can be overridden by higher brain function
		\end{itemize}
\end{itemize}






\end{document}
