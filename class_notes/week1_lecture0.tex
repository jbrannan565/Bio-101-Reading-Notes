\documentclass{article}
\usepackage[utf8]{inputenc}

\title{Week 1 Lecture 0: ``lecture 3 - Characteristics of Life"}
\author{Jared Brannan }

\usepackage{natbib}
\usepackage{graphicx}
\usepackage{mathtools}
\usepackage{amsthm}
\usepackage{amsmath}
\usepackage{amssymb}
\usepackage{xcolor}
\usepackage{bbm}
\usepackage{bm}
\usepackage{physics}

\theoremstyle{definition}

\newcommand{\upRiemannint}[2]{
\overline{\int_{#1}^{#2}}
}
\newcommand{\loRiemannint}[2]{
\underline{\int_{#1}^{#2}}
}

\newtheorem{definition}{Definition}
\newtheorem{asside}{Asside}
\newtheorem{conjecture}{Conjecture}
\newtheorem{example}{Example}
\newtheorem{theorem}{Theorem}
\newtheorem{lemma}{Lemma}
\newtheorem{puzzle}{Puzzle}
\newtheorem{corollary}{Corollary}
\newtheorem{proposition}{Proposition}


\begin{document}
\maketitle

\section{Announcments and other drivel}

List of scientists is now available on Canvas.\\~\\

Since the '70s the number of insects as dropped by at least 40\%, which is a sign of ecological collapse. There are also signs that the amazon rainforest is nearing ecological collapse, which will result in 10s of thousands of species going extinct, and have a great effect on the processes that regulate atmospheric composition.\\~\\

\section{Characteristics of life}
How do we know if something is living?

E.g. Is the carona virus a living organism (at least by cell theory)? No. Viruses seem to be ``on the border" of what we would call life.

\subsection{What is life?}
So, what are the characteristics? What is life? The broad categories:

- Response to external stimuli

- Adapt to the enviornment

- Contain materials found only in living organisms - Some protiens and molecules are unique to living things.

- Alter the environment. E.g. Beavers making dams

- Acquiree and use energy

- Maintain a constant internal environment - \textit{homeostasis}. Organisms alter their internal systems to keep things like temperature, ph, chemical composition, etc. constant with environmental changes.

- Sense the environment. E.g. we don't sense all of the environment, for example the infrared spectrum is invisible to us. Insects without water in their eyes, however, can see. We can detect UV by the pain we feel from sun burns.

- Reproduce. This is a hallmark of living things. Most organisms have a limitted lifespan, and so to carry their genes forward organisms must make some sort of copy of themselves.

- Have a high degree of organization - Each type of organism has an organization to its structure that heavily regulates its function.

\subsubsection{Sense and response}

At different levels: 

- \textbf{Nervous system} (most animals have one)

\indent\indent* note, single cell organisms \textit{don't} have nervous systems.

- \textbf{Individual level}: coordination of organ systems

- \textbf{population level} (long term): w/ evolution

- \textbf{Ecology} (interactions)

\subsubsection{Structure}

All organisms have a particular body structure, and unique protiens. Protiens can be shared across species, but usually with subtle variation.

\subsubsection{Organization}

Chemicals $\implies$ cellular $\implies$ tissues $\implies$ organs $\implies$ organ systems $\implies$ organisms

\subsubsection{Reproduction}
\indent\indent- Most organisms have a reproductive system, which is seperated (usually) into male and female individuals.

- Ther are processes of pregnancy and developement.
\indent\indent * things go from a small fertilized egg, into a large, complex, multicellular structure.

- Important constituants include genetics, inheritance, and evolution.

- reproduction is the way organisms transmit their properties to their offspring.

- there are high incentives for organisms to have a strong reproductive drive.

\subsubsection{Homeostasis}
Most living systems engage in homeostasis - etemology: 

homeo = similar to, equal, 

stasis = standing still.

So \textit{homeostasis} means to stay the same.

- Cellular function and life are only possible under fairly speciic environmental conditions for each species (temp, mousture, salinity, atmospheric pressure)
\indent\indent * some critters match their temp to their environment, but mamals set their own (to the best of their ability)

- The environment can have major influences on physiological performance.

\indent\indent * E.g. plants in the desert need mechanisms to extract lots of moisture out of the  air and soil, while in the tropics, plants don't need to be so desperate.\\~\\

\textbf{Examples of homeostatic systems in humans:}

Heat ballance: 

- core body temp - 98.6F among humans (recent work shows 97.9F may be closer)

\indent\indent* above ~109 is lethal

\indent\indent* bellow ~90 is an emergency, belloow ~80 is lethal

Energy ballance:

- blood sugar - too high, and you are diabetic

- blood oxygen

Water ballance:

- blood pressure

- blood water volume - if we're dehydrated, we shrink in size

- there is a narrow range of internal conditions that allow propper bodily funcntion

- homeostasis is the tendency to maintain that range

- but external environments change, meaning the body has to adjust

\end{document}
