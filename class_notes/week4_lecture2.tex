\documentclass{article}
\usepackage[utf8]{inputenc}

\title{Week 4 Lecture 2}
\author{Jared Brannan }

\usepackage{natbib}
\usepackage{graphicx}
\usepackage{mathtools}
\usepackage{amsthm}
\usepackage{amsmath}
\usepackage{amssymb}
\usepackage{xcolor}
\usepackage{bbm}
\usepackage{bm}
\usepackage{physics}

% indent first line
\usepackage{indentfirst}

\theoremstyle{definition}

\newcommand{\upRiemannint}[2]{
\overline{\int_{#1}^{#2}}
}
\newcommand{\loRiemannint}[2]{
\underline{\int_{#1}^{#2}}
}

\newtheorem{definition}{Definition}
\newtheorem{asside}{Asside}
\newtheorem{conjecture}{Conjecture}
\newtheorem{example}{Example}
\newtheorem{theorem}{Theorem}
\newtheorem{lemma}{Lemma}
\newtheorem{puzzle}{Puzzle}
\newtheorem{corollary}{Corollary}
\newtheorem{proposition}{Proposition}


\begin{document}
\maketitle

\section{Administrative drivel}
\begin{itemize}
	\item Class avg on exam: about $53\%$.
	\item Exam will be passed back probably Monday.
	\item First paper submission is next Friday.
\end{itemize}

\section{More on nucleic acids...}
\begin{itemize}
	\item  Nucleic acids are used for energy storage: ATP
	\item This is for energy \textit{transfer,}  not really storage
	\item ATP: Adenosine Tri-Phosphate
		\begin{itemize}
			\item energy is stored in the bonds (between 3 phosphate groups (alpha,beta, gamma))
			\item breaking off the alpha group releases the energy by adding water, resulting in a Phosphate and Adenosine diphosphate (ADP)
				\begin{itemize}
					\item H2O + ATP = P + ADP + ENERGY
				\end{itemize}
			\item The reverse process builds ATP
				\begin{itemize}
					\item ADP + P + ENRGY = H2O + ATP
				\end{itemize}
			\item these 2 processes are carried out by enzymes
			\item ``money of matabolism"
			\item has the same basic structure of a nuclic acid
			\item sometimes the beta bond is broken, into AMP (Adenosine monophosphate), but much less often
		\end{itemize}
	\item There is a process for breaking down glucose, and trapping that energy into the bonds of ATP
	\item clicker q: Of which polymer is a gene composed? DNA
\end{itemize}

Thus concludes the organic molecules.

\section{Cells}
Clicker q: What biological polymer is between transcription and translation? RNA

Polymer: a chemical structure made up of many repeating molecules (poly-mer means "many repeating units")

\subsection{Membranes}

\begin{itemize}
	\item Basic structure: Phospholipid bilayer
	\item Front (anterior) back (posterior)
	\item cells undergo cell division to multiply by splitting
	\item shows the example of a roundworm cell deviding into the full oranism c-elegans
	\item endoplasmic reticulum: where RNA is turned into proteins
	\item Cells are structured!
		\begin{itemize}
			\item phospholipid bilayer (see those notes)
		\end{itemize}
	\item cells are water based!
	\item inter/intracellular liquid is mostly water
	\item see  picture of cell
	\item All of the structures (organelles) in a cell are membrane bound in eukareotes (not so in prokareotes!)
		\begin{itemize}
			\item organelle = "little organ"
		\end{itemize}
	\item In the nucleus, DNA translated into RNA in DNA, RNA translated into protiens in the endoplasmic reticulum
	\item The Mitochondria makes ATP
	\item Solutions:
		\begin{itemize}
			\item \textbf{Medium/solvent} (e.g. water, and is the main one)
				\begin{itemize}
					\item water allows biochemistry to occur in most situations, except for lipids, which need help (from enzymes?)
				\end{itemize}
			\item \textbf{Solute} (e.g. salt, sugar, proteins, ions, hormones, etc)
			\item \textbf{Concentrationn} = (amount of dissolved stuff) / volume
			\item Within the solution, particles move at RANDOM
				\begin{itemize}
					\item This porcess of random movement is called diffusion
					\item diffusion is good for transmitting molecules over short distances, but not over long distances
					\item any organism bigger than a few centimeters need extra structures to overcome this
				\end{itemize}
		\end{itemize}
	\item Diffusion:
		\begin{itemize}
			\item If particles are clumped in one area, their random movement will eventually distribute them evenly inn the water (this is called \textbf{Diffusion})
			\item partices move from high concentration to low concentration
		\end{itemize}
	\item Passive diffusion:
		\begin{itemize}
			\item The phospholipid bilayer is permiable to some small molecules (most don't)
				\begin{itemize}
					\item only really small things can get through
					\item e.g. CO2
					\item Fat soluble things can also pass through!
				\end{itemize}
			\item If there is a concentration gradient between the inside and outside of the membrane, molecules will pass through the membrane to get the consentration to be even using diffusion.
			\item Since the membrane isn't all that permiable, the concentration gradient is never really even
		\end{itemize}
\end{itemize}


\end{document}
