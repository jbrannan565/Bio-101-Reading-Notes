\documentclass{article}
\usepackage[utf8]{inputenc}

\title{Week 6 Lecture 1}
\author{Jared Brannan }

\usepackage{natbib}
\usepackage{graphicx}
\usepackage{mathtools}
\usepackage{amsthm}
\usepackage{amsmath}
\usepackage{amssymb}
\usepackage{xcolor}
\usepackage{bbm}
\usepackage{bm}
\usepackage{physics}

% indent first line
\usepackage{indentfirst}
% one inch margins
\usepackage[margin=1.0in]{geometry}

\theoremstyle{definition}

\newcommand{\upRiemannint}[2]{
\overline{\int_{#1}^{#2}}
}
\newcommand{\loRiemannint}[2]{
\underline{\int_{#1}^{#2}}
}

\newtheorem{definition}{Definition}
\newtheorem{asside}{Asside}
\newtheorem{conjecture}{Conjecture}
\newtheorem{example}{Example}
\newtheorem{theorem}{Theorem}
\newtheorem{lemma}{Lemma}
\newtheorem{puzzle}{Puzzle}
\newtheorem{corollary}{Corollary}
\newtheorem{proposition}{Proposition}


\begin{document}
\maketitle

\section{Administrative drivel}
\begin{itemize}
	\item Exam next friday. Covering from stuff just after exam 1 till what we get to on Wednesday.
\end{itemize}

\section{Anatomy and Physiology}
\subsubsection{Bones}
\begin{itemize}
	\item Repair:
	\item Left off talking about what happens when you break a bone.
		\begin{itemize}
			\item It's been found that bones can be lengthened by breaking and allowing it to begin to regrow, then pull the end further apart and repeat.
		\end{itemize}
	\item Biomechanics:
		\begin{itemize}
			\item Bones have neat properties, cause they're strong
			\item Bones are moderately good at tension, great at compression, and TERRIBLE at torsion
			\item Tensile forces pull the bone from both ends, but usually modest amounts of force (body weight of individual)
			\item Compressive forces tend to be much greater than tensile forces.
			\item A femur can take 17k lbs of compressivev force!
			\item Bone is made up of calcium phosphate used to make hydroxylapatite, collagen
			\item calcium minerals are very strong under compesive forces, and crumbles under torsion and tension
			\item Collagen is strong under tension and torsion, and bends under compression
			\item bone is a composite material, making it very strong
		\end{itemize}
	\item Bone health -- exercise and nutritionn
		\begin{itemize}
			\item Since bone is a living tissue, it responds to activity
			\item so, if it undergoes stronger forces, it will become stronger (from bone repair)
			\item Daily activity (mechanical stress) keeps bones at the level required for the acctivity
			\item Sedentary lifestyle leads too less bone density
				\begin{itemize}
					\item more porous (more spongy bone)
					\item more brittle
				\end{itemize}
			\item Active live leads to more bone deensity
			\item Calcium and vitamin D intake is essential for bone health
			\item Calcium tipically comes from meat and leafy green vegetables
			\item Bone loss start to occur after around age 50, ,more exaggerated in females once menopause has occured, wmen are more vunerable to osteoporosis,  = moroe fragile bones, increased resk of fracture / damaga in old age
			\item Bones density continues to increase till 20s, then steady out till 50s, then steadily declines
			\item excess calcium is excreted in the urine
		\end{itemize}
	\item cartilege and collagen are not the same thing, but they are similar
	\item clicker q: what to osteoclasts do? Dissolve bone minerals
	\item end of bones
\end{itemize}

\subsubsection{Muscles}
\begin{itemize}
	\item  Muscles are contractile tissues -- the change their length
	\item kinds: smooth, cardiac, and skeletal (striated)
	\item here we'll look at skeletal muscle tissue
	\item Antagonism pair
		\begin{itemize}
			\item Opposite sets of muscles result in opposite motion -- called extensor/flexor pairs, e.g. tricps/bisseps
			\item each pulls other back to full length
			\item \textbf{Tendons}  connect muscles to bone (accross a joint)
			\item \textbf{Ligaments} are bone-bone connections
			\item tendons and ligaments are similar, be tendons have extra things like nerves that communicate to the nervous system how much force is going on
			\item mmuscles can only contract, so they can only pull, not push
		\end{itemize}
	\item structure of muscle
		\begin{itemize}
			\item Most of the  action of contraction happens at a microscopic level
			\item tendon on one end
			\item Theere's a sheath around a bundle of fibres, which makes up the muscle which is made up of fibers, each of which is a muscle cell
			\item muscle cells are multicelluar, they start as many, then fuse into the long strand
			\item in between the fibers are mitocondria
			\item there's a sarcoplasmic reticulum corresponding too each strand, and is similar to the smooth ER; critical in muscle contraction
			\item Each muscle cell is made up of long tubes called myobibril, which is made up of many repeating segments.
			\item each segment is called a sarcomere, which is the part of the muscle that actually contracts!
			\item two protiens that do the contraction:
				\begin{itemize}
					\item Thin filament (actin) -- looks like beads on a string (the string is the actin), 2 twisted together, attached to the ends of the sarcomere
					\item Thick filament (myosin) -- looks like a golf club, all stacked side by side 
				\end{itemize}
			\item the more sacromeres stacked side by side, the stronger, the longer the stack end by end, the faster the contraction
		\end{itemize}
	\item Clicker q: what is a sarcomere? A unit of contraction
	\item 
\end{itemize}



\end{document}
