\documentclass{article}
\usepackage[utf8]{inputenc}

\title{Week 13 Lecture 2}
\author{Jared Brannan }

\usepackage{natbib}
\usepackage{graphicx}
\graphicspath{ {./figures/} }
\usepackage{mathtools}
\usepackage{amsthm}
\usepackage{amsmath}
\usepackage{amssymb}
\usepackage{xcolor}
\usepackage{bbm}
\usepackage{bm}
\usepackage{physics}

% indent first line
\usepackage{indentfirst}
% one inch margins
\usepackage[margin=1.0in]{geometry}

\theoremstyle{definition}

\newcommand{\upRiemannint}[2]{
\overline{\int_{#1}^{#2}}
}
\newcommand{\loRiemannint}[2]{
\underline{\int_{#1}^{#2}}
}

\newtheorem{definition}{Definition}
\newtheorem{asside}{Asside}
\newtheorem{conjecture}{Conjecture}
\newtheorem{example}{Example}
\newtheorem{theorem}{Theorem}
\newtheorem{lemma}{Lemma}
\newtheorem{puzzle}{Puzzle}
\newtheorem{corollary}{Corollary}
\newtheorem{proposition}{Proposition}


\begin{document}
\maketitle

\section{Administrative drivel}
\begin{itemize}
	\item Exam booklets are here to pickup
		\begin{itemize}
			\item mean 50.5, curve 24.5
		\end{itemize}
	\item grade to date will be updated over the weekend to account for the new exam!
\end{itemize}

\section{Digestive system}
\begin{itemize}
	\item Two parts: (review)
		\begin{itemize}
			\item GI tract
				\begin{itemize}
					\item main tube
				\end{itemize}
			\item Accessory organs
		\end{itemize}
	\item Processes:
		\begin{itemize}
			\item ingestion -- get food into the body (gut)
				\begin{itemize}
					\item put it in your mouth
				\end{itemize}
			\item mixing -- mix food around and move it through tract
			\item digestion -- mechanical and chemical break-down
				\begin{itemize}
					\item e.g. mechanical: stomach churning, chewing
					\item e.g. chemical: converting things into molecules that can be used
				\end{itemize}
			\item absorption -- pass food molecules into the body
				\begin{itemize}
					\item molecules pass through the walls of the GI tract into the blood or lymphatic system to distribute the new molecules for use throughout the body
				\end{itemize}
			\item defecation -- get rid of anything left over
				\begin{itemize}
					\item poop.
				\end{itemize}
		\end{itemize}
	\item Organ overview -- check canvas for picture!
		\begin{itemize}
			\item the red labeled things are accessory organs, the black labeled things are GI tract organs
		\end{itemize}
	\item Ingestion -- theeth:
		\begin{itemize}
			\item phyiscal breakdown of food -- bring food into the mouth/oral cavity and mastication (grinding the foods, in other words, chewing)
			\item large items are torn apart by the incisors to make smaller peices
		\end{itemize}
	\item Ingestions -- saliva:
		\begin{itemize}
			\item salivary glands are considered accessory organs
			\item lubricate food
				\begin{itemize}
					\item saliva is slippery
					\item also, adds moisture so it's more of a paste
				\end{itemize}
			\item Enzymes  start metabolism:
				\begin{itemize}
					\item salivary \textbf{amylase} for carbs (starch, glycogen)
					\item salivary lipase for lipids 
						\begin{itemize}
							\item only really active in infants
						\end{itemize}
				\end{itemize}
			\item lysozyme targets bacteria
				\begin{itemize}
					\item covered in the immune section
					\item lots of bacteria on the food we injest
				\end{itemize}
		\end{itemize}
	\item Ingestion -- swallowing
		\begin{itemize}
			\item Tongue moves to the roof of the mouth, forcing food into the esophogus where more muscles push the food down into the stomach
			\item at the back of the throat, the trachea and esophegus meet.
				\begin{itemize}
					\item when you swallow, there's a valve at the top of the trachea that closes by the food forcing it closed
						\begin{itemize}
							\item valve == epiglotis
						\end{itemize}
					\item choking is when this goes wrong
				\end{itemize}
		\end{itemize}
	\item GI tract -- \textbf{peristalsis}  == wavelike muscle movements that push food along
		\begin{itemize}
			\item all parts of the GI tract do this
		\end{itemize}
	\item Esophagus
		\begin{itemize}
			\item Tube
			\item \textbf{Functions}: transfers food from oral cavity/mouth to stomach
			\item about 2 feet long
		\end{itemize}
	\item Stopmach
		\begin{itemize}
			\item Expandable sac with an opening at either end
			\item Sphincter -- ring of muscles that acts like a valve, controls movement from one section of GI tract to next
			\item 2 sphincters:
				\begin{itemize}
					\item Esophageal -- at the esophogus
						\begin{itemize}
							\item prevents stomach acid from getting into the esophagus
						\end{itemize}
					\item pyloric -- at the small intesstine
						\begin{itemize}
							\item prevents acid from getting into the intestines
							\item regulates the rate of food entering the intestine
						\end{itemize}
				\end{itemize}
			\item \textbf{Functions}: 1) store and mix food, 2) break down large molecules (chemical), 3) destroy invaders, 4) regulate release of food into intestine
				\begin{itemize}
					\item stores food for several hours, allowing for intermitant activity between eating
					\item this is convinient, since it was once dangerous to go out and hunt for food
					\item for us now, it just allows us to work or play between meals
					\item pepsin along with acid breaks down food
						\begin{itemize}
							\item most of digestion doesn't occur in the stomach
							\item pepsin breaks down some protiens
						\end{itemize}
					\item amylase breakdown of carbs stops upon entering the stomach
					\item the most important function is its high acidity
						\begin{itemize}
							\item kills most pathogens
						\end{itemize}
				\end{itemize}
			\item "Gastric juices" == 2-3 liters/day
				\begin{itemize}
					\item produce protease to digest protiens
					\item Very acidic: pH = 2
						\begin{itemize}
							\item each intiger increase is 10 times the previous integer
						\end{itemize}
					\item Low pH helps kill microorganisms
				\end{itemize}
			\item Food + gastric juices = thin, watery -- \textbf{Chyme} 
			\item about a foot long
		\end{itemize}
	\item clicker q: where in the GI tract is the first place chemical digestion of food occurs? the mouth (amylase breaking down carbs)
	\item stomach -- mixing and release:
		\begin{itemize}
			\item waves every (about) 15 seconds slosh CHYME around
				\begin{itemize}
					\item passes a small amount of chyme to the small intestine
				\end{itemize}
		\end{itemize}
	\item small intestine
		\begin{itemize}
			\item one of the longest sections of the GI tract
			\item on the order of 18 feet
			\item parts: Duodenum, jejunum, ileum
			\item Function: 1) Chemical digestion and 2) absorb nutrients, especially macro ones
				\begin{itemize}
					\item most chemical digestion happens in the Duodenum (first foot)
				\end{itemize}
		\end{itemize}
	\item Duodenum:
		\begin{itemize}
			\item receives bile from the liver (stored in the gal blader) and enzymes from the pancreas
			\item pancreatic secretions also neutralize low pH of chyme exiting the stomach
			\item chyme needs to be neutralized before it can be chemically digested, otherwise proteins will denature
				\begin{itemize}
					\item this is done with bicarbonate
				\end{itemize}
			\item RE Wierd Al's song about the pancreas: flow flow flow pancreatic juice, flow flow, into the duodenum
		\end{itemize}
	\item Pancreas
		\begin{itemize}
			\item secrete enzymes that digest macronutrients:
				\begin{itemize}
					\item carbs broken by -- sucrase and amylase
					\item proteins broken by -- trypsin
					\item lipids broken by -- lipase
				\end{itemize}
		\end{itemize}
	\item Liver and gallbladder
		\begin{itemize}
			\item \textbf{Bile} -- produced by the liver as a byproduct of hemoglobin and cholesteroll breakdown and stored in the gallblader
				\begin{itemize}
					\item attacks lipids
				\end{itemize}
			\item Bile salts released into duodenum when high lipid content detected
			\item Bile emulsifies fats, makes the molecules accessible to lipase
				\begin{itemize}
					\item rate of lipase break down of lipids is a function of the surface area to volume ratio of fat chunks, so smaller chunks are broken down faster
				\end{itemize}
		\end{itemize}
\end{itemize}
							
\end{document}
