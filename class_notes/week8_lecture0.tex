\documentclass{article}
\usepackage[utf8]{inputenc}

\title{Week 8 Lecture 0}
\author{Jared Brannan }

\usepackage{natbib}
\usepackage{graphicx}
\graphicspath{ {./figures/} }
\usepackage{mathtools}
\usepackage{amsthm}
\usepackage{amsmath}
\usepackage{amssymb}
\usepackage{xcolor}
\usepackage{bbm}
\usepackage{bm}
\usepackage{physics}

% indent first line
\usepackage{indentfirst}
% one inch margins
\usepackage[margin=1.0in]{geometry}

\theoremstyle{definition}

\newcommand{\upRiemannint}[2]{
\overline{\int_{#1}^{#2}}
}
\newcommand{\loRiemannint}[2]{
\underline{\int_{#1}^{#2}}
}

\newtheorem{definition}{Definition}
\newtheorem{asside}{Asside}
\newtheorem{conjecture}{Conjecture}
\newtheorem{example}{Example}
\newtheorem{theorem}{Theorem}
\newtheorem{lemma}{Lemma}
\newtheorem{puzzle}{Puzzle}
\newtheorem{corollary}{Corollary}
\newtheorem{proposition}{Proposition}


\begin{document}
\maketitle

\section{Administrative drivel}
\begin{itemize}
	\item Exams haven't been looked at -- still a few makeups to be done
		\begin{itemize}
			\item Could be a week or 2 before we get grades...
		\end{itemize}
\end{itemize}

\section{Anatomy and Physiology}

\subsection{Cardiovascular system continued...}
\subsection{Heart}
\subsubsection{Blood pressure}
\begin{itemize}
	\item \textit{Blood pressure}  is what moves the blood through the blood vessels
		\begin{itemize}
			\item pressure generated by the heart with ventrical compression
			\item ventricles contract, forcing more blood out into the vessels, increasing pressure
		\end{itemize}
	\item when ventricles contract == \textbf{Systolic} 
	\item when ventricles relaxe == \textbf{Diastoic} 
	\item BP usually 120 / 80 (in milimeters of mercury) (140 is dangerous!) (120 == systolic pressure, 80 == diastolic)
		\begin{itemize}
			\item BP = ratio of your blood pressure when ventricles are contracted and when the heart is relaxed.
		\end{itemize}
	\item at high blood pressure vessels can begin to fail
	\item The \textbf{radial pulse}  is felt on the wrist, just under the tumb
		\begin{itemize}
			\item You can feel the blood pressure changing
		\end{itemize}
	\item The \textbf{carotid pulse} is felt:
		\begin{itemize}
			\item lateral to the trachea
			\item meedial to the sternocleidomastoid muscle
		\end{itemize}
	\item most pulses are in 65-100 bpm range
	\item In general, lower HR means a more efficient heart and better cardiovascular fitness
	\item lots of things effect HR
		\begin{itemize}
			\item activity level
			\item air temp
			\item Standing/sitting/laying
			\item emotions and stress
			\item body size
			\item medications
			\item diet
		\end{itemize}
\end{itemize}

\subsection{Blood}
\begin{itemize}
	\item Red blood cells carry $O_2$ (and a small portion of the $CO_2$)
	\item white blood cells  handle immune function
		\begin{itemize}
			\item motile
			\item if there's an infection, they can leave the blood vessels
		\end{itemize}
	\item platelets govern blood clotting
		\begin{itemize}
			\item lots of protiens are involved too
		\end{itemize}
	\item plasma is the fluid that carries dissolved particles and wastes (carries most of the $CO_2)$
	\item blood is a  connective tissue
	\item the arteries pulse too to maintain the pressure on the blood to get it to the capilaries
	\item Red blood cells:
		\begin{itemize}
			\item Each cell carries around 200 million hemooglobin molecules
			\item Each hemoglobin has a Hem, which is an iron atom that binds the oxygen, and can bind 4 oxygen
			\item so each cell can carry about 800 million oxygen
			\item no nucleus, so theres a dimple, only in mamals
		\end{itemize}
	\item Blood clotting:
		\begin{itemize}
			\item Platelets (small peices of red blood cells)
				\begin{itemize}
					\item red blood cells get old, and some are recycled into platlets among other things (eg bile)
				\end{itemize}
			\item fibrin, fibrinogen (clotting protien)
				\begin{itemize}
					\item forms a mesh over the injury and catch platlets which fill the holes, making it harder for blood to get through
					\item then a scab forms and the capilaries that were broken are rebuilt
				\end{itemize}
			\item Hemopheliacs can't clot
		\end{itemize}
	\item The volume of the blood in humans is about 5 liters (plasma + cells)
		\begin{itemize}
			\item If this was evenly distributed over the body you would faint
		\end{itemize}
	\item \textbf{Plasma}
		\begin{itemize}
			\item makes up $55\%$ of the blook
			\item water, dissolved prooteins, electrolytes (salt ions), glucose, hormones, $CO_2$, clotting factors.
		\end{itemize}
	\item clicker q: What do platelets do? Help blood clot
\end{itemize}

\subsection{Blood vessels}
\begin{itemize}
	\item Arteries, veins, capilaries
	\item arteris get narrower as they move away from the heart
	\item Veins get larger as they move toward the heart
	\item these differences in diameter helps maintain blood pressure
	\item blood is moving from high to low pressure at all times
	\item eventualy, thin enough arteris are called artiriols and they reach the capilaries
	\item the capilaries in total length is massiive
	\item Artieris:
		\begin{itemize}
			\item Carry oxygenated blood from the heart to the systemic tissues; deoxygenated (somewhat) blood to the lungs
			\item Thicker, more muscle -- helps move blood towards capillaries in the tissues
			\item touch connective tissue on outside to prevent too much expansion
				\begin{itemize}
					\item sometimes this fails, and the artery will burst, leading to internal bleeding
				\end{itemize}
			\item 2 layers of smooth muscle to maintain blood pressure
			\item has an epithelial layer allowing blood to moove through with low friction
		\end{itemize}
	\item Veins:
		\begin{itemize}
			\item Carry deoxygenated blood back to the heart from the systemic tissues; oxygenated blood from the lungs
			\item don't have much muscle, but they have valves!
		\end{itemize}
\end{itemize}


\end{document}
