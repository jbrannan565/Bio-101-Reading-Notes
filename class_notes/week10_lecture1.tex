\documentclass{article}
\usepackage[utf8]{inputenc}

\title{Week 10 Lecture 1}
\author{Jared Brannan }

\usepackage{natbib}
\usepackage{graphicx}
\graphicspath{ {./figures/} }
\usepackage{mathtools}
\usepackage{amsthm}
\usepackage{amsmath}
\usepackage{amssymb}
\usepackage{xcolor}
\usepackage{bbm}
\usepackage{bm}
\usepackage{physics}

% indent first line
\usepackage{indentfirst}
% one inch margins
\usepackage[margin=1.0in]{geometry}

\theoremstyle{definition}

\newcommand{\upRiemannint}[2]{
\overline{\int_{#1}^{#2}}
}
\newcommand{\loRiemannint}[2]{
\underline{\int_{#1}^{#2}}
}

\newtheorem{definition}{Definition}
\newtheorem{asside}{Asside}
\newtheorem{conjecture}{Conjecture}
\newtheorem{example}{Example}
\newtheorem{theorem}{Theorem}
\newtheorem{lemma}{Lemma}
\newtheorem{puzzle}{Puzzle}
\newtheorem{corollary}{Corollary}
\newtheorem{proposition}{Proposition}


\begin{document}
\maketitle

\section{Administrative drivel}
\begin{itemize}
	\item  Last day to pick up exams
	\item Lecture images through respiration are posted
	\item The review sheet through respiration has been posted -- to be updated up till wednesday
	\item Those with 100\% on the term papers, submit without revision
\end{itemize}

\section{Defence and repair -- immune system}
\begin{itemize}
	\item \textbf{Chemical barriers}  -- review
		\begin{itemize}
			\item Oils
			\item  Salts
			\item Saliva -- lysozyme
				\begin{itemize}
					\item Dodgs have a lot of lysozyme, so getting licked can be beneficial
				\end{itemize}
		\end{itemize}
	\item \textbf{More chemical barriers} 
		\begin{itemize}
			\item \textbf{Acid}  -- secreted by the stomach
				\begin{itemize}
					\item Hydrochloric acid chemically rips apart  pathogens
						\begin{itemize}
							\item Cells in the walls produce this acid
							\item also assists with digestion
							\item Allows pepsidogen to be broken into pepsin -- beginning of protien digestion, then finished in the small/large intestin
							\item the stomach also mechanically breaks the food down
							\item The stomach mainly functions for the storage of food (so you don't have to eat constantly exposing you to danger), and 
							\item the high acid level kills contaminants in food
						\end{itemize}
					\item pH = 2 (between lemon juice and battery acid)
				\end{itemize}
			\item \textbf{Helpers}  -- beneficial bacteria and fungi -- ``probiotics"
				\begin{itemize}
					\item Large populations in the gut outcompete pathogens
						\begin{itemize}
							\item makes it harder for outsiders to get a foothold
						\end{itemize}
					\item Glycogen secreted within the vagina promotes \textit{Lactobacillus}  bacteria, which consume glycogen and convert it back to lactic acid
				\end{itemize}
		\end{itemize}
	\item \textbf{Oh no}  -- the barrier is breached
		\begin{itemize}
			\item Damage to the skin give pathogens acces to your insides
			\item Need to builid the walls back up
		\end{itemize}
	\item Wound repair: -- what needs to happen
		\begin{itemize}
			\item Stop blood loss
				\begin{itemize}
					\item Clotting does this
				\end{itemize}
			\item Remove damaged and dead cells 
			\item Destroy any pathogens that got in
				\begin{itemize}
					\item battle.
				\end{itemize}
			\item Reconstruct the barrier
				\begin{itemize}
					\item epidermis grows back in along with some scar tissue
				\end{itemize}
		\end{itemize}
	\item Stop blood loss:
		\begin{itemize}
			\item 1. \textbf{Vasoconstriction} 
				\begin{itemize}
					\item min continuing blood loss
				\end{itemize}
			\item 2 \textbf{Platelet}  activation
				\begin{itemize}
					\item Platelets stick to the injury site
					\item platelets change shape, becoming sticky
					\item platelets trigger clotting protiens
					\item clotting proteins activate \textbf{fbrin}  to form a net over the wound
				\end{itemize}
			\item 3. \textbf{Clot formation} 
				\begin{itemize}
					\item Red blood cells get caught in the net 
					\item this clot (scab) plgs the wound, blocking further blood loss
				\end{itemize}
		\end{itemize}
	\item Remove damaged and dead cells
		\begin{itemize}
			\item \underline{Inflammatory response} brings in clean-up cells
				\begin{itemize}
					\item mostly white blood cells
					\item fight infection and remove damaged or dead cells
				\end{itemize}
		\end{itemize}
	\item Inflamation:
		\begin{itemize}
			\item 1. Damaged cells release chemical messengers
			\item 2. this signals other cells (mast cells) which send a response (histamines)
			\item 3. Which leads to vasodialation (the sweeling of the capilaries)
				\begin{itemize}
					\item This allows larger gaps in the wall to form, allowing white blood cells and fluid into the surrounding tissue to do battle
					\item The fluid that gets out of the capilaries causes swelling
				\end{itemize}
			\item 4. More signals from the damaged cells triger Phagoccytes
				\begin{itemize}
					\item macrophages, neutrophils -- white blood cells
				\end{itemize}
		\end{itemize}
	\item \textbf{Phagocytes:} -- type of white blood cell
		\begin{itemize}
			\item \textbf{Macrophages and neutrophils}  (phagocytes) can get out of the capillaries near the injury site and into the damaged tissue -- they can move on their own!
				\begin{itemize}
					\item each attack and kill in specific ways
					\item macrophages are the most numerous, and engulf many pathogens into an internal vesicle, bring digestive enzymes in a seperate vesicle to them, and digest them
					\item neutrophils are suicide bombers, ingulfing the pathogen and killing themselves, releasing chemicals that kill nearby cells
				\end{itemize}
			\item 1. Mast cells cells detect injury to nearby cells and release histamine, initiating inflamatory response
			\item 2. Histamine increases blood flow to the wound sites, bringing in phagoocytes and other immune cells that neutralize pathogens. 
			\item Phagocytes die, turning into pus
			\item pimple == dead phagocytes after fighting a bacterial infection of an oil gland in a hair folicle
		\end{itemize}
	\item Bacteria are evereywhere in the environment, so generally some enter in a wound, thus, usually some to kill off.
	\item Reconstruct the barrier
		\begin{itemize}
			\item Reconnect the vasculature
				\begin{itemize}
					\item remaining capillaries grow toward the wound site, and toward each other
				\end{itemize}
			\item Rebuild the physical structure
				\begin{itemize}
					\item epidermal cells reconnect under the scab
					\item extracellular collagen protein gets deposited into the wound site under the epidermis
					\item repair collagen is slightly different in structure and arrangement than normal skin collagen -- \textbf{scar} 
				\end{itemize}
		\end{itemize}
\end{itemize}

\end{document}
