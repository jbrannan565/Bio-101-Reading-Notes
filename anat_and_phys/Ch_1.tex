\documentclass{article}
\usepackage[utf8]{inputenc}

\title{An Introduction to the Human Body}
\author{Jared Brannan }

\usepackage{natbib}
\usepackage{graphicx}
\usepackage{mathtools}
\usepackage{amsthm}
\usepackage{amsmath}
\usepackage{amssymb}
\usepackage{xcolor}
\usepackage{bbm}
\usepackage{bm}
\usepackage{physics}
\usepackage{hyperref}

\theoremstyle{definition}

\newcommand{\upRiemannint}[2]{
\overline{\int_{#1}^{#2}}
}
\newcommand{\loRiemannint}[2]{
\underline{\int_{#1}^{#2}}
}

\newtheorem{definition}{Definition}
\newtheorem{asside}{Asside}
\newtheorem{conjecture}{Conjecture}
\newtheorem{example}{Example}
\newtheorem{theorem}{Theorem}
\newtheorem{lemma}{Lemma}
\newtheorem{puzzle}{Puzzle}
\newtheorem{corollary}{Corollary}
\newtheorem{proposition}{Proposition}


\begin{document}
\maketitle

\textbf{Disclaimer:} at the moment these notes are a summary of the chapter review and need to be filled in with more specific information.

\\~\\

\textbf{Chapter Objectives}
\indent- Distinguish between anatomy and physiology, and identify several branches of each.

- Describe the structure of the body, from simplest to most complex, in terms of the six levels of organization

- Identify the functional characteristics of human life

- Identify the four requirements for human survival

- Define homeostasis and explain its importance to normal human functioning

- Use appropriate anatomical terms to identify key body structures, body regions, and direcctions in the body

- Compare and contrast at least four medical imaging techniques in terms of their function and use in medicine

\section{Overview of Anatomy and Physiology}
\definition \textbf{Human anatomy}
is the scientific study of the body's structures. This includes chemistry and physics within bodily structures.

It used to be studied by analyzing wounds, then but dissection of \textit{cadavers} (human corpses), now we use imaging techniques to look into living bodies.


\definition \textit{Physiology}
aims to explain how bodily structures cooperate to maintain life.

These two topics are closely related an inform one another.

\section{Structural Organization of the Human Body}

Life of the human body is maintained at many structural levels: chemical, cellular, tissue, organ, organ, system, and the full organism (and more). 

Higher levels are built on the lower ones. (see \href{https://github.com/jbrannan565/Bio-101-Reading-Notes/blob/main/bio_2e/Ch_1.pdf}{notes} on Chapter 1 of the Biology OpenStax text)

\section{Functions of Human Life}
The control of processes within a human body is not simulataneous, and are carried out continuously to build, maintain and sustain life. 

The included processes:

- organizationn, in terms of maintenance of essential body boundaries

- metabolism, including energy transfer via anabolic and catabolic reactions

- responsiveness

- movement

- growth

- differentiation

- reproduction

- renewal

\section{Requirements for Human Life}

The key resources for human survival:

- oxygen, at least every few minutes

- water, at least every several days

- carbohydrates, lipids, proteins, vitamins, minerals, at least every several weeks.

- moderate temperature

- precise atmospheric pressure (keeps gases in solution within the body. E.g. oxygen in the blood)

- High enough blood pressure (ensures that nutrients get to vital organs, and such.) but low enough to not damage blood vesels.

\section{Homeostasis}

\definition \textit{Homeostasis}: the activity of cells that maintains their state in a fassion that sustains the life of the organism is called \textit{homeeostasis}.

- These processes consist of possitive (intensify response to stimulus) and negative (prevent excessive response to stimulus) feedback loops (mostly negative), with components

1. stimulus

2. sensor

3. control center

4. effector

\section{Anatomical Terms}

\indent- Built on latin words

- Directional/positional terms exist

1. precise positional words. E.g. ``occipital" refers to the back of the head

2. directional terms. E.g. anterior and posterior

3. spliting the body into three planes: the sagittal, frontal and transverse

4. the interior has two main cavities: dorsal and ventral (also refered to by the respective directional terms in 2.), which have sub cavities.

5. Serous membranes, two layers: parietal and visceral, which surround a fluid filled space and cover the lungs (pleural serosa), heart (pericardial serosa), and some abdominopelvic organs (peritoneal serosa).

\section{Medical Imaging}

\indent- Used to be done with hand drawings (15th/16th centuries)

- now we use X-Rays, CT scans, MRI scans, PET scans, and ultrasonography, which give a detailed image of the internal structures of living bodies

\end{document}
