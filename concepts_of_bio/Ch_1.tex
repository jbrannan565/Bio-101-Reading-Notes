\documentclass{article}
\usepackage[utf8]{inputenc}

\title{Chapter 1: The Study of Life}
\author{Jared Brannan }

\usepackage{natbib}
\usepackage{graphicx}
\usepackage{mathtools}
\usepackage{amsthm}
\usepackage{amsmath}
\usepackage{amssymb}
\usepackage{xcolor}
\usepackage{bbm}
\usepackage{bm}
\usepackage{physics}

\theoremstyle{definition}

\newcommand{\upRiemannint}[2]{
\overline{\int_{#1}^{#2}}
}
\newcommand{\loRiemannint}[2]{
\underline{\int_{#1}^{#2}}
}

\newtheorem{definition}{Definition}
\newtheorem{asside}{Asside}
\newtheorem{conjecture}{Conjecture}
\newtheorem{example}{Example}
\newtheorem{theorem}{Theorem}
\newtheorem{lemma}{Lemma}
\newtheorem{puzzle}{Puzzle}
\newtheorem{corollary}{Corollary}
\newtheorem{proposition}{Proposition}


\begin{document}
\maketitle

These are the notes to chapter 1 of the "Biology 2e" text, which contains almost the same exact content as chapter 1 of Concepts of Biology (including figures...).

\section{The Science of Biology}

Section goals:
- Id similarities btw natural sciences
- summarize scientific method
- compare inductive and deductive reasoning
- Desc the goals of basic sciec and applieed science.

\definition \textit{Biology} is simply the study of life.

\subsection{The process of Science}

\definition \textit{science}
knowledge that covers general truths or the operation of general laws, especially when acquired and tested by the scientific method.

\definition \textit{scientific method}
a method of research with defined steps that include experiments and careful observation.

\definition \textit{hypothesis}
is a suggested explanation for an event which one can test.

Note: some sciences do not test by expiriment, but instead look for evidence for or against a hypothesis (E.g. archeology).

\definition \textit{theory}
a tested and confirmed explanation for observations or phenomena.

\subsection{Natural sciences}
\definition \textit{natural sciences}
Scientific fields that relate to the physical world and its phenomena and processes.
E.g.  astronomy, biology, chemistry, earth science, and physics.

Other divisions:
\definition \textit{life sciences}
the study of living things (includes biologoy)

\definition \textit{physical sciences}
the study of nonliving matter (E.g. astronomy, physics, chemistry)

Some disciplines fit both categories and are called \textit{interdisciplinary}.
E.g. biochemistry, biophysics.

There are hard and soft(er) sciences: hard refering to sciences that use quantitative data, and soft(er) refering to sciences that use more qualitative data.

\subsection{Scientific Reasoning}
Here we focus on what it means \textit{to know}. To gain knowledge, scientists rely on two modes of thinking: inductive reasoning and deductive reasoning.

\definition \textit{Inductive reasoning}
a form of logical thinking that uses related observations to arrive at a general conclusion. In this realm, observations increase or decrease the likelyhood of a conclusion.

\definition \textit{Deductive reasoning}
a form of logical thinking that uses a general principle or law to forecast a specific result.

Deductive reasoning is concerned with binary states (the truth or falsity of claims), while inductive reasoning is more concerned with the \textit{likelyhood} of something being true or false.

Two modes of discovery:

\definition \textit{Descriptive (or discovery) science}
aims to observe, explore and discover (usually using inductive reasoning).

\definition \textit{hypothesis based science}
begins with a specific question or problem and a potential answer that can be tests (relys mostly on deductive reasoning).

\subsection{The Scientific Method}
First documented by Sir Francis Bacon (1561-1626). The following are the loose steps which most sciences follow (sometimes in a "looping" manner, going back to previous steps) which we call the \textit{scientific method}: a useful process for making discoveries about the natural world.

The pet example provided in the text is that of a warm room and a student askes "Why is the classroom warm".

\subsubsection{Proposing a Hypothesis}
\indent- We can propose several hypothese. E.g.

\indent\indent - "The classroom is warm because no one turned on the AC."

\indent\indent	- "The classroom is warm because there is a power failure, and so the AC doesn't work."

- After a hypothesis is selected, a prediction can be made in the form "if ... then ..." E.g.
\indent\indent - "If the student turns on the AC, then the classroom will no longer be too warm"

\subsubsection{Testing a Hypothesis}
Hypothesese must be \textit{testable}. Hence they must be

\definition \textit{Falsifiable}
a claim that can be shown to be false (or disproven) by experiment is said to be \textit{falsifiable}.

Note: science doesn't \textit{prove} things, but instead disproves or increases our confidence in a claim.

E.g. supernatural claims are \textit{not} falsifiable, usually.

Experiments that test a hypothesis have variables and controls, are split into multiple \textit{experimental groups} (seperate groups to be tested within an experiment).

\definition \textit{variable} any part of the experiment that can vary or change during the experiment.

\definition \textit{control group} an expiremental group where manipulations from the hypothesis are not carried out.


the scientific method is then given by the following steps:

1. Make an observation

2. Ask a question

3. Form a hypothesis that answers that question

4. Make a prediction based on that hypothesis

5. Do an experiment to test the prediction

6. Analyze results

\indent\indent- If hypothesis is supported, go to 7.

\indent\indent- If hypothesis is not supported, go back to 3. and try again.

7. Report the results

\subsection{Two types of science: Basic Science and Applied Science}

\definition \textit{Basic science} or "pure" science seeks to expand knowledge reegardless of the short-term application of that knowledge. It is not focused on providing immediate value. "Knowledge for knowledge sake".

\definition \textit{Applied science} or "technology" aims to use science to solve real-world problems.

\definition \textit{serendipity} knowledge that is acquired by a "happy accident".

\subsection{Reporting Scientific Work}
Research needs to be reported so peers can be aware of new knowleddge. This is done in

\definition \textit{Peer-reviewed manuscripts} scientific papers that a scientist's colleagues or peers review.

\subsubsection{Parts of a scientific paper}

1. \textit{Abstract} - a concise summary at the begining of the paper. Could include an outline

2. \textit{Introduction} - starts with a brief, but broad, backround on what is know in the field that the paper is within. Usually contains reasoning for the work being done. Refers to published work of other scientists. 

3. \textit{Materials and methods} - includes a complete and accurate description of the methods, techniques and substances used to gather data. The intent of this section is to allow other researchers to reproduce the results.

4. \textit{Results} - a narration of findings without interpretation, usually including data

5. \textit{Discussion} - interpretation of results, description of the relationships between variables, and explanations of observations. Usually, other researchers work is cited here.

6. \textit{Conlcusion} - summarizes important experimental findings. May (usually) contain future directions for the work.

An accronym for this is \textbf{IMRAD}. Note, these sections usually won't be found in \textit{review articles}, which are secondary papers that comment on the state of a field.

\section{Themes and Concepts of Biology}
Goals of the section:

- ID and describe properties of life

- Desc the level of organization among living things

- Recognize and interpret a phylogenetic tree

- list examples of different subdisciplines in biology

\subsection{Properties of life}

Shared Characteristics or functions of living organisms: sensitivity or response to the environment,
reproduction, adaptation, growth and development, regulation/homeostasis, energy processing, and evolution (all of which will be covered in detail in later sections). These 8 properties together define life.

\subsubsection{Order}
Organisms are highly organized, coordinated structures.

Structure of cellular organisms: atoms make up molecules which make up oorganells and other cellular inclusions.

Multicellular organisms are made up of \textit{tissues} (collections of like cells), which make up \textit{organs} (body structures that serve a distinct function), which make up organ systems.

\subsubsection{Sensitivity of Response to Stimuli}

Life (organisms) respond when stimulated (go figure).

\subsubsection{Reproduction}

\definition \textit{Reproduction}
an organism's ability to copy itself. E.g. DNA replication and cell division for single celled organisms, sexual reproduction for multicellular organisms.

\subsubsection{Adaptation}

\definition \textit{adaptation}
an organisms ability to fit an enviornment. i.e. the ability to change (either as an individual, or generationaly through evolution by natural selection) to better suite itself to its enviornment.

Adaptations serve to increase the likelyhood of reproduction, and are not constant.

\subsubsection{Growth and Development}

An organisms genes lay out instructions for their cells to grow and develop, resulting in individuals having similar traits to their parents.

\subsubsection{Regulation/Homeostasis}

\definition \textit{Homeostasis} - steady state -
an organisms ability to maintain internal conditions with a narrow range, despite  enviornmental changes.

\subsubsection{Energy Processing}

Organisms have the ability to process sources of energy, and use this energy during metabolic processes.

\subsubsection{Evoution}
Since this is much more specific than the above, I'll provide the direct quote:

``The diversity of life on Earth is a result of mutations, or random changes in hereditary material over time. These mutations allow
the possibility for organisms to adapt to a changing environment. An organism that evolves characteristics fit for the
environment will have greater reproductive success, subject to the forces of natural selection."

\subsection{Levels of Organization of Living  Things}

There's a heirarchy from small to large among living things:

- \textit{Atom} - small unit of matter is a nucleus surrounded by electron clouds

- \textit{Molecule} - a chemical structure made up of at least two atoms connected via chemical bonds

- \textit{Macromolecules} - large molecules usually constructed via polymerization (monomers are combined into a polymer). E.g. DNA

- \textit{Organelles} - macromolecules surrounded by membranes. Are usually found within cells.

- \textit{cell} - an organisms smallest fundamental unit of structure and function

- \textit{multicellular organisms} - two kinds

\indent\indent1. \textit{Prokaryotes} - single celled or colonial organisms without membran-bound nuclei

\indent\indent2. \textit{Eukaryotes} - like 1., but have membrane-bound organelles and nuclei.

- \textit{tissues} - combinations of cells in larger organisms typpically made up of cells with similar function.

- \textit{organs} - collections of tissues that perform one function.

- \textit{organ system} - combinations of functionally related organs

- \textit{organism} - individual living entities. e.i. individuals

- \textit{population} - individuals of a species within a specific area

- \textit{community} - all inhabitants within an area

- \textit{ecosystem} - all living things in an area, along with abiotic factors (non-living things that contribute to the community or popualation in any way)

- \textit{biosphere} -  the collection of all ecosystems. The zones of life on earth.

\subsection{The Diversity of Life}

The broad scope of biology is due to the wide variety (diversity) of life, which is due to \textbf{evolution}.

\definition \textit{Phylogenetic tree} - a summary of the relationships between species given by connecting lines  where their ansestries connect. (easily seen from figure 1.7 in the text).

\subsection{Branches of Biological Study}

Biology has many branches. E.g.:

\textit{Melcular biology and biochemistry} study molecules and chemicals related to biology, like interactions between RNA, DNA, and protiens and their regualtion.

\textit{Microbiology} study microorganisms. i.e. single-cell organisms.

\textit{Neurobiology} - the study of nervous systems

\textit{Paleontolog} - the study of life's history using fossiles

\textit{Zoology and botany} - the study of plants and animals respectively


\end{document}
